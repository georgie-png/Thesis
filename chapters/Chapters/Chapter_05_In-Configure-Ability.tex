\hypertarget{in-configure-ability}{%
\section[In-configure-ability]{\texorpdfstring{\protect\hypertarget{anchor}{}{}In-configure-ability}{In-configure-ability}}\label{in-configure-ability}}

In this Chapter I reflect on the intersectional inquiry I undertook into
Configure-Able methods as a member of In-grid. This inquiry builds out
of In-grid's slowly sedimented collective practices and aims to
highlight how through first person action research and disobedient
action research methods I have introduced critical access as a critical
framework and practice to reorient our ways of relating to and
coalescing tables of social and technical infrastructure together. It
specifically orients how we formed technical practices of collective
access that disoriented the norms of expertise within our network and
organisational practices and their knowledges. In this I aim to share
how Crip theory and critical access doesn't have to be extra time or
assimilationist, but can be a place to collectively orient from in the
configuration of our communities and their infrastructures.

To do this I am following of from prior chapters forming of
\href{../../04_Configure-able_Methods/04_Configure-Able\%20Methods.md}{04\_Configure-Able
Methods}, I aim to go beyond that initial ``definition'' and move to
feel it out in action and through collective inquiry. I do this by
initially setting a background for these figures to emerge from, acting
as a ground for In-grid's practices to be contoured and figured out
from. I then reflect on a focus group I organised for In-grid members
that reflected on the longer arcs in our collective practices through a
framing of critical access. Here I made room for us to figure out
frictions in these experiences and in doing so questioning the
inflexibilities of the systems we felt. Orienting from these points we
questioned how we had formed wiggle room around or within these points
of impact through In-flex-ability. This is a recurring word play from
us, where In-grid get In- to all sorts of trouble. In-flex-ability as we
go on to define through action is manifested through interdependence in
action, and where we can make room together to distribute and move
around these systemic pressures and frictions, aiming to move towards
how we want to collectively orient ourselves. This framing of
In-flex-ability is also meant to move these dialogues beyond the
problem/solution axis of curative design practices, and instead focus on
how In-grid address and orient towards not only validating our feeling
around these existing frictions, but make room for that feeling as part
of the sense making and figuring out processes of configuration.

Working through this framing, I reflect on In-grids collective inquiry,
as well as my first person inquiry, into how we approached these
systemic inflexibilities through our collective in-flex-abilities. These
inquiries then go on to take into closer consideration the technical
network practices we undertook together. This inquiry resulted in the
forming of accessible technical for the Servpub and Tinc VPN that I
reflect on in
\href{05.03.01.00_Configure-ability_In-Docs.md}{05.03.01.00\_Configure-ability\_In-Docs}.
From these we also emerged a set of workshops to make more space for
these collective figuring out practices that we ran at 4S/EASST 2024 as
well as internally, which I reflect on in
\href{05.03.02.00_Configure-ability_In-Workshops.md}{05.03.02.00\_Configure-ability\_In-Workshops}.
From these In-flex-able technical network practise I reflect on the
counter figures and configurations we manifested along the way, as well
as the emergence into In-grid's first Feminist Server Manifesto, that we
called Femfester and published with Artists Running Datacentres (Simms
et al., 2024).

\hypertarget{background}{%
\subsection[Background]{\texorpdfstring{\protect\hypertarget{anchor}{}{}Background}{Background}}\label{background}}

To form the figures of this inquiry I will first start by forming the
background for them to emerge from, and trace the long tail that builds
up to contour these bodies. In-grid's background rises from our
emergence during a collective residency run at Arebyte Gallery in 2020
at the start of the pandemic. This was run by Rachel Falconner, Helen
Pritchard and Rabecca Edwards. During this residency a group of 10-20 of
us formed a set of skill shares and collaborations with artists and
technical practitioners across a number of fields. This was held mostly
online, and focused on people's fluctuating capacities during a very
stressful and unpredictable pandemic, forming a very specific relation
for working. We often joke about how this has trauma bonded us together.

Since the residency we kept communing together, talking and keeping in
touch. It was a year or so later when the pandemic and lock down started
to ease that we began to come back together and regularly work again.
This was during the first year of my PhD and during this time we built
up multiple collective events, forming distributed ways of working and
facilitating outcomes together. These ranged from a series of club
nights/events called \emph{Real Bodies} beginning at Corsica nightclub,
with many installed works and performances, to \emph{Chain} at Iklectic
where In-grid formed a preformative infrastructure that evolved over the
night. Since then we have kept working together forming other
exhibitions, infrastructures and skill shares collectively, as well as
slowly establishing our internal organisation practices in the small
bits of time we can be together.

Over this time people have joined and left, so we have balanced out to
be a core group of 5-8 people and max 10-15 who can join in and
contribute where they want and are able. These blury numbers may seem
odd but they are due to our open door policy, which makes rooms for
members to come and go as they need. This structure has formed through
us orienting to make space for those of limited capacities due to work,
life or health, as well as our constant lack of (institutional) funding
that is common in community/arts work in post austerity UK. We also have
a wide range of disciplines, capacities and approaches, from musicians
to web developers and community activists. This brings an eclectic set
of goal and desires to the background of this group.

Another focus in this background is the 8M Trans*feminist counter cloud
strike\footnote{https://www.in-grid.io/projects/8m/} that we took a part
of in 2021 and the years since with a number of different collectives in
coalition. This meeting of many communities and collectives was also a
major shift in In-grids desires and background. This strike focused on
moving away from big tech ecologies and instead starting to invest our
time, energy and resources into re-approaching how we can work with and
for technical infrastructures towards our own collective orientation of
them. These meetings formed the basis for our desire to not only
question In-grid's own collective infrastructure but to also start to
approach being in coalition with other communities.

From this background I crop into two different scales of inquiry for
this research. Within these inquiries and reflections I aim to both
demonstrate Configure-Able Methods in action, and how they are localised
within and emerge from In-grid's dynamic and intersectional background.

\hypertarget{figuring-out-inflexibility}{%
\subsubsection[Figuring out
Inflexibility]{\texorpdfstring{\protect\hypertarget{anchor}{}{}Figuring
out
Inflexibility}{Figuring out Inflexibility}}\label{figuring-out-inflexibility}}

When defining inflexibility, I proposed a few questions and prompts for
us to work through together and to make room for us to define these
terms by reflecting on our experiences of our collective practice. With
these questions around inflexibility we touched upon many different
aspects of what this term could offer to us. One of the major points of
disorientation within our dialogues and definition of inflexibility was
when I brought crip studies into this conversation. This was after
members of In-grid joked about capitalism being the ultimate
inflexibility, forcing our bodies into uncomfortable and unwanted
orientations just to get by. In response I agreed, bringing this jest
into focus with crip studies opening up the ways it critiques
capitalism's invalidation of bodies and matters through dialogues of
productivity, value and normalisation. From this opening up of crip
studies and critical access, we very much overflowed with experiences
and critiques of the frictions we had felt together. One clear example
we touched upon from reflecting on In-grids collective organisational
practice is how banking systems make no room for organisations like us,
yet we need a bank to work legally and be valid. This dynamic in
practice left us not only with a very limited choice of one bank to work
with, but the one that remained did not really match our politics. This
process of getting that misfitting bank also took us months to sort out
through their dysfunctional admin system, and our part-time
non-expertise capacities. Here we made sense of these frictions and how
these systems had taken us from how we wished we could orient ourselves
(e.g.~having easy flow of funds held by an ethical source) to the
inflexible position of having a glitchy bank account profiteered off by
unethical businesses.

One point of friction around the definition of inflex came from whether
restraints we put on ourselves could be understood as inflexibility. For
example whether us keeping accountable to one another, taking our time
and refusing certain relations or politics would be us being inflexible.
Here I turned us again to this relational critique of access from crip
studies, where the ``problem'' is not held in our individual bodies or
needs being met, but rather in the social and systemic relations of
politics that limit our ability to reach our needs that could otherwise
be possible. With this notion inflexibility is positioned within the
systems and their relations, in it needing us to perform specific roles
(users), and with specific bodily figures and horizons (normids), to be
able to be valid and approachable on their sedimented often inaccessible
line. Reflecting on this later on though we do touch upon how In-grid,
as our members work in institutes and were taught by them, have these
systems and politics embedded in us, and how we have tried to make room
to work through these frictions we feel in ourselves.

\hypertarget{defining-inflexibility}{%
\subsubsection[Defining
Inflexibility]{\texorpdfstring{\protect\hypertarget{anchor}{}{}Defining
Inflexibility}{Defining Inflexibility}}\label{defining-inflexibility}}

When defining inflexibility, I proposed a few questions for us to work
through together and to help define these terms and reflect through our
collective practices and experiences. With these questions around
inflexibility we touched upon many different aspects of what this term
could offer to us. One of the major points of reorienting within our
dialogues and definition of inflexibility was when I brought crip
studies into this conversation. This was after members of In-grid joked
about capitalism being the ultimate inflexibility, forcing our bodies
into uncomfortable and unwanted orientations just to get by. In response
I agreed, bringing this jest into focus with crip studies opening up the
ways it critiques capitalism's invalidation of bodies and matters
through dialogues of productivity, value and normalisation. From this
opening up of crip studies and critical access, we very much overflowed
with experiences and critiques of the frictions we had felt together.
One clear example we touched upon from reflecting on In-grids collective
organisational practice is how banking systems make no room for
organisations like us, yet we need a bank to work easily, legally and
viably. This dynamic left us with not only a very limited choice of one
bank to work with, and the one that remain did not really match our
politics, but also that this process has taken months to sort out
through their dysfunctional admin system. Here we made sense of these
frictions and how these systems had taken us from how we wished we could
orient ourselves (e.g.~having easy flow of funds held by an ethical
source) to the inflexible position of having a glitchy bank account
profiteered off by unethical businesses.

One point of friction around the definition of inflex came from whether
restraints we put on ourselves could be understood as inflexibility. For
example whether us keeping accountable to one another, taking our time
and refusing certain relations or politics would be us being inflexible.
Here I turned us again to this relational critique of access from crip
studies, where the ``problem'' is not held in our individual practices
of meeting our needs, but rather in the systemic relations of politics
that limit our ability to reach our needs that could otherwise be
possible. With this notion inflexibility is positioned always within the
system, in it needing us to perform specific roles (users), and with
specific bodily figures and horizons (normids), to be able to be valid
and approachable.

This point of where we went out of our ways intentionally together, and
stepped out of line to orient towards our needs, politics and practices
was in fact what I was hoping to discuss as in-flex-ability. It is the
times we have wiggled to make room in these norms for the flexibility a
body like our needs to be present.

\hypertarget{figuring-out-in-flex-ability}{%
\subsubsection[Figuring out
In-flex-ability]{\texorpdfstring{\protect\hypertarget{anchor}{}{}Figuring
out
In-flex-ability}{Figuring out In-flex-ability}}\label{figuring-out-in-flex-ability}}

When it came to defining In-flex-ability in a similar way, and where I
had prepared some prompts and questions, we instead decided to orient to
figuring it out in inquire and action. Here we reflected on how we
improvised other ways of approaching these places of friction within the
sedimented norms we felt friction with. As a group we decided on which
practice's frictions and misfittings we wanted to orient from and in
which ways we wanted to approach them. This resulted in us making sense
of in-flex-ability by reflecting on the inflexibilty we felt in our
processes and practices of \emph{De/re-clouding}, \emph{Coalition and
knowledge exchange} and our \emph{In-ternal politics}. When doing these
it took us a minute to warm up to this framing of
inflexibility/in-flex-ability, but through this emerged some interesting
points of collective reflection.

\hypertarget{dere-clouding}{%
\paragraph[De/re-clouding]{\texorpdfstring{\protect\hypertarget{anchor}{}{}De/re-clouding}{De/re-clouding}}\label{dere-clouding}}

This was our first topic and opened up to us to discus for a while as we
felt out this framing of analysis. This topic's scope focused on our
long term retreat from the norms of hosting our collective
infrastructures with big tech cloud, to instead orient towards community
and artist run cloud. This movement started after we had joined the 8M
communities mentioned in the background of the chapter. During this
transition we have slowly moved from a scattering of big tech cloud
services to a focused collaboration of our cloud and network
infrastructures held with Servus.

Some of the inflexibility and frictions we felt when moving out of and
retreating from the norms of Big Tech network infrastructures were how
as a product/service which is divided into user/designer relations, it
meant that we felt defined to a specific role and body with no agency.
To do this these Big Tech cloud infrastructures often put people out of
reach of the matteriality of what they are using, and aim to keep the
user experience, and the shaping of the human factor friction-less. To
keep these infrastructures even more out of reach the sedimented norms
and defaults of sysadmin and network maintenance is oriented as a
burden, joyless and friction-full. It is also often only done by the
isolated expert working alone or from a distance to configure technical
relations for other users who are strangers to them.

To form wiggle room here specifically we practised in-flex-ability
through interdependence and collaboration. In this moving our scoping
from needing to take on this immense burden that is out of reach of us,
to instead ask what we have capacity do as In-grid, and what we can rely
on others for. For instance, we as a group could not host our cloud
infrastructures due to member capacity, experience and knowledges as
well as a number of other reasons, so are working with Servus to host
them. In this relation forming wiggle room around the frictionles user
roles we were forced into by big tech and instead through the frictions
of working with Servus, gently coming into contact with and making sense
of the material frictions of network infrastructures, our needs within
them and how we want to orient towards them. On an experimental server
we have hosted with them we have started to form more wiggle room here
by starting to host our own VPN and distributed network, as well as a
git code repository (we are working towards). This orienting of scales,
capacities and scope towards interdependence and coalition made room for
us to not only move off of the big tech cloud, but also towards building
up relations with similar communities and putting these capacities in
reach of our collective futures.

\hypertarget{working-groups-and-knowledge-exchange}{%
\paragraph[Working groups and knowledge
exchange]{\texorpdfstring{\protect\hypertarget{anchor}{}{}Working groups
and knowledge
exchange}{Working groups and knowledge exchange}}\label{working-groups-and-knowledge-exchange}}

For this focus we reflected on how we as a collective have worked with
others and exchanged knowledge, practices and built relations along the
way. This work is very much emerging for In-grid, both in the sense that
we are only just beginning to work with other collectives and
communities but also in that we are starting to do more things and
splinter off into smaller working groups. Both of these were places
where we felt we needed to make room for feedback and exchange within
the wider group for these practices and knowledges.

The inflexibility and friction we felt here came from two points of
impact. One was how we relates as a group and with others, and how these
exchanges are handled. Here we reflected back to points where relations
had induced anxiety and had a huge impact on our capacity for exchange.
These for us often originated form relations where we felt like others
were trying to catch people out, penalise them, or not open up relations
on a personnel situated level without prescribing them to a role. This
was mainly felt when working with other collaborators, but of course
still existed internally. We felt that these sorts of sedimented
relations of policing undermined not only our capacities to work
together as we needed to constantly validate our position, but also our
enthusiasm and energy for those collaborations. ==The other friction we
return to is our lack of funding, and how this curtails our capacity to
meet the many people and matters we have in relation with the intimacy
we wish we could.==

To care for, feel out and wiggle room within these inflexibilities,
In-grid reflected on how we try our softest to have patience and be
generous with each other and other others. This for us meant taking time
to share knowledge, make room to let it sink in, bringing clarity
through repetition, not in a way to reinforce bodies and relations, but
like a tic (Maier et al., 2020) when someone forgets or is lost we
gently re-orient ourselves together. So when someones like ``why are we
even bothering with banking/cloud/coalitions/being here it seems so much
trouble'', we can again go over why we have chosen to put ourselves in
awkward relations to practice in-flex-ability around these systemic
Inflexibilities, but are doing this in a way that distributes and
reforms these sedimented relations. It is when someone vulnerably turns
up 2 months later and they need us to recap some of that time to be
in-sync-ish. For In-grid it was also about figuring out ways of making
these repetitions and in-flex-abilities joy-full even. This is still
very much an ongoing dialogue and one where the many members of In-grid
are slowly forming our own practices and politics that can refuse these
penal norms we live among, and form wiggle room and social/systemic
flexibility for people to bring their own ways of being together in our
collaborations.

\hypertarget{in-ternal-politics}{%
\paragraph[In-ternal
politics]{\texorpdfstring{\protect\hypertarget{anchor}{}{}In-ternal
politics}{In-ternal politics}}\label{in-ternal-politics}}

Here In-grid turned in-ward to frictions within how we as a group figure
out and orient our own collective politics in practice. This focus
offered us room to reflect on how we as a group form goals and orient
towards them. In this dialogue we also made room to question what those
goals currently are. It also has to be said that this was definitely the
most fun topic to chat about, with lots of joy and giggles, when
reflecting on these impact of these frictions and our resulting care for
them.

The inflexibility we felt here orients a number of matters. This is
explicit when we question what is possible with limited funding, what we
can squeeze in around work and what (if any) of our activities are
understood as valid and valued within these systems. We also touched
upon how we as students of these academic institutions, teachers and
researchers within them, as well as within the norms of other
computing/creative industries have inherited and sedimented the politics
and norms of these bodies within us. Here we ask how these norms
reinforce us to be inflexible in certain ways so that the contingent
politics and relations we orient towards are out of reach. One example
we experienced and raised around this came from both academic
institutional norms, as well as ones of communities around us, and which
oriented towards the ease of centring knowledge in individuals as it was
more efficient than dispersing it into collective knowledges. This
centring of practises and their knowledges though situates all the
capacity to do certain things within certain people. Here by not taking
on this troublesome task of distributing these practices and their
knowledges the organisation can loose these capacities easily in doing
so, through certain people leaving, becoming ill/mad/dissociative or
passing. Beyond this consolidating these practices and their knowledges
into an individual also takes them out of reach of others, and makes no
room for them to know those practices and politics as a group, or the
plurality knowings that makes room for.

In-response we reflected on how In-grid made room within and around
these normalised institutionalised practices of management and
collaboration that we have experienced. Becky and Katie surfaced a joke
they have here about how if anyone needed to understand In-grid's
methods and process, they should read the \emph{Simple Sabotage Field
Manual} (1944) by the CIA. This is not true in every way, we are not out
to break our own things, like this manual suggest for taking down
fascist states and their infrastructures. Instead when approaching
collective practice In-grid is slow and we take our time, we are
dedicated to working in large groups\footnote{``Attempt to make the
  committees as large as possible - never less than five'' (CIA, 1944,
  p.~28)}, have many misunderstandings\footnote{``Misunderstand''
  orders. Ask endless questions or engage in long correspondence about
  such orders." (CIA, 1944, p.~29)}, and lots of emotions to
share\footnote{``Cry and sob hysterically at every occasion, especially
  when confronted by government clerks.'' (CIA, 1944, p.~32)}. These
actions for us, and maybe because many in the manual were appropriated
from anti-fascist, communist and anarchist movements are the many
practices we have to come to decisions, build trust and get consensus
from different members on our orientations. The many ways we orient this
differently to the CIA manual of course is that the actions we take are
done through love and care, where we aim to build up In-grid's
relations, instead of weaponizing them. In many ways this shares how we
do break down efficient management systems, but In-grids orients towards
other ways of being together instead of replacing one fascist state with
another. It is only through years of these relations building up that we
have emerged the trust that makes room for members to take different
approaches, forget what we were doing, and to be okay with being
uncomfortable. For example getting our cloud setup moved to Servus took
over a year to get everyone set up with account on it and for us to feel
like we as a group knew what we are doing there. Still though there are
some members who don't want to be on there and that is fine, and we have
found ways to be in-flex-able around these points of friction so to not
excluded them. In these ticing practices we build trust and patience in
letting people manage their own way to reflect on and learn from our
collective practices and their frictions. This demonstrates well our
practice of in-flex-ability, and shares how we have wiggled room
together within the sedimented and inherited norms for the flexibility
In-grids member's need and manifest to be together within these
inflexible systems and times.

\hypertarget{processing-in-configure-ability}{%
\subsection[Processing
In-Configure-ability]{\texorpdfstring{\protect\hypertarget{anchor}{}{}Processing
In-Configure-ability}{Processing In-Configure-ability}}\label{processing-in-configure-ability}}

This section offers up a collective inquiry and reflection by In-grid
around our longer term practices and where we processed together how
Configure-able methods has manifested locally in our context. To do this
I initiated a focus group and working session called \emph{In-fra
processing} to reflect back over these years of work together with
In-grid members through the framing of Configure-able methods. This
became a place for us to make room around/within and to trouble the
frictions and inflexibilities we felt from our experiences of
configuring these systems through our collective practices of
Configure-Ability. This focus group was also run as part of the annual
In-grid In-ternal residency to make room for us to query and reorient
our approaches together as we went forward. This session was run as a
very relaxed online call where we got together and took collective notes
and reflections on this
\href{https://digitalcare.noho.st/pad/p/Infra_processing}{pad}. Here we
initially started by figuring out some of the inflexibility we had felt
in practice and from them recollected our collective practices of
in-flex-ability in practise. Doing this we aimed to frame our
configure-able methods and practices of in-flex-ability by making-sense
of how we approached and reoriented the norms of these relations we came
in contact with.

\hypertarget{configure-ability-in-}{%
\subsection[Configure-ability
In-]{\texorpdfstring{\protect\hypertarget{anchor}{}{}Configure-ability
In-}{Configure-ability In-}}\label{configure-ability-in-}}

These next two sections take on a closer inquiry into how Configure-able
methods have emerged through In-grid's network infrastructuring
practices. Our enactment of the methods here approach how these
technical practices and their knowledges are shared. I initially
approach this inquiry through first person action research reflections
around our collective technical and documentation practices. I then
follow this up with a in-ternal In-grid workshop as an example of how
these methods have manifested both accessible social and technical
practises to make room for groups to disorient and figure out network
infrastructure as we configure them.

\hypertarget{old-workshops}{%
\subsection[old (
workshops???)]{\texorpdfstring{\protect\hypertarget{anchor}{}{}old (
workshops???)}{old ( workshops???)}}\label{old-workshops}}

This was both in the formation of the technical docs, but especially
within the ``practising protocols'' workshops that emerged from them.
These workshops focused on making room for people to collectively
configure out these technical practices from their own experiences,
contexts and situated knowledges. These workshops were originally
developed to offer people accessibly insight into our collective
practices through (con)figuring out infrastructures together and was
presented as part a combined panel at 4S/EASST 2024 that we
co-organised. This focus of this inquiry here though is the internal
In-grid workshop of ``practising protocols'', where we collectively
(con)figured out our new server together. I focus though on when I
brought these workshops internally to enable us to set up our first
collective server and in doing so manifesting intentions for it and how
we want to work and relate through and around it. This was collated into
a manifesto for our first collectively written publication, called the
Femfester Manifesto (Simms et al., 2024) for the artists running data
centres journal. Through these set of actions I aim to reflect on how
these workshops have emerged as a way for us to make accessible not only
technical practices of network infrastructuring, but also enable a
flex-ability and agency to (con)figure them out politically and
relationally otherwise in context together. \#\#\# Configure-ability
In-Docs

A lot of this specific practice is currently being written up on the
\href{https://wiki4print.servpub.net/index.php?title=Chapter_3:_Praxis_Doubling}{Servpub
Wikic4Print}, and orients these reflections through a collaborative text
alongside other members of In-grid, which this thesis restraints do not
make room for. Here though I want to surface how in collaboration with
In-grid, and with my critical input from this research we collectively
tried to query the norms of technical practice, and their docs through
critical access. By taking up critical access here we are both asking
how can we make room for these practices and their knowledges to be more
accessible to wider groups and dialogues through the ways we describe
and format this information, but also how this can make space for people
to dispute, unsediment and disorient together the network
infrastructures we have inherited.

Carrying on the framing from the
\href{05.02_Processing\%20In-Configure-ability.md}{05.02\_Processing
In-Configure-ability} section, where In-grid orients approaching
systemic inflexibility through our collective practice of
In-flex-ability, I share how we have started to make room around these
norms.

\hypertarget{inflexibility}{%
\paragraph[Inflexibility]{\texorpdfstring{\protect\hypertarget{anchor}{}{}Inflexibility}{Inflexibility}}\label{inflexibility}}

Here I initiate this inquiry by examining the inflexibilities we found
both within the social and technical aspects when making the docs for
the Servpub infrastructure. When the sub working team of
In-grid\footnote{Katie Tindle, Batool Desouky, Suni Lao, Rebecca Aston +
  Me. \#\# (Con)figuring In-frastructures} initially tried to figure out
the practices necessary for the Servpub infrastructure from other
communities within the collaboration and the official docs of the VPN
(TINC\footnote{https://tinc-vpn.org/ \#\#\#\# in-flex-ability}), we
found it inaccessible and full of barriers for a few reasons. We are a
group of fairly technically capable people, and even though this was a
new technical practice for us, it was hard to get to grips with what was
going on at this table of network infrastructures. In reflection this
was in many places due to the technical documentation and resources we
were recommended, and which were quite abstract or incomplete. Beyond
this the time/capacity In-grid had was very limited as it was mostly
voluntary work at this stage and so it was a struggle to reach this
dictated outcomes. The workshops and working sessions explaining the
practice were also intensive and long sessions of terminal/console work,
which often over ran, leaving everyone tired, discombobulated and unable
to document or remember what had happened clearly. The formats of both
felt fairly pre-configured and very rigid and did not make space for
In-grid to have agency or input our background, approaches and ways of
working around these sorts of frictions. Instead we were told ``why
can't we be like others'', aka be quite and listen to/reproduce the
``one best way'' of feminist server practices.

Looking closer at the materiality of sysadmin as a technical process and
practice it is quite specific in the ways it is made to be knowable.
This practice for me personally has taken time to firstly understand and
get used to the how the terminal environment works, finding handy
commands, short keys and approaches that are absent from the very
minimal void of an interface. The norms for this sort of terminal based
sysadmin practice from my experience are written technical documentation
which aim to act as references while you work. As part of our practice
into this environment we started to make our own technical docs and as
we came into contact with them started to question the norms of these
sedimented approaches to configuring network infrastructures. From these
sites of impact we started to question how these doc formats made these
practices know-able and perform-able in very specific ways and by
specific people. Technical docs normally take on the form of text and
code intervals, which often procedurally get you to run commands, whilst
giving you little background knowledge into what is being done or why.
Some are even less descriptive and just have code functions and commands
abstractly explained in technical language and figures which are
inaccessible to non-expert practitioners. The background knowledge is
itself often isolated to what needs to be known to run that command,
setup that software or execute that code and does not clearly state
politics or orientations of the software/infrastructures background.
With Hamraie's flexible user (2017) we can understand that these norms
are highly inflexible and the expert/super user is a shaping of a very
specific body and person that most cannot access. The norms of these
formats and the expected users they make room for hold the promise or
(political) intentions of a specific setup, pipeline or approach towards
network infrastructuring. Being procedural unquestioning formats they
dictate specific relations, politics and figures to manifest your local
infrastructure through. This can be understood as taking on the
determined roles and operational metaphors of a system without capacity
to question them, or be given space to localise, improvise and
reinterpret them to your own relations, desires and needs. To me this
made me reflect on the ways that Miriyam Aouragh and Paula Chakravartty
in \emph{Infrastructures of Empire} (2016) discuss how infrastructures
bring with them and reinforce the politics of the creator, and with this
there is a need to question these inherited norms and make room for this
locality when manifesting them. These norms of docs and their technical
practices hold these infrastructures in space, in line and for most out
of reach.

Another key inflexibility of the Serpub infrastructure and these docs
was the maintainability of its setup. This setup was predominantly based
around the use of Tinc, which even though a very accessible VPN
(compared to others) and one that comes from a genealogy of autonomous
feminist servers, was becoming deprecated. This meant in many ways that
any infrastructure based on these docs would (even if maintained) become
obsolete, insecure and dysfunctional. This restraint to maintenance and
deprecation crosses into all docs and infrastructures, but here provides
a point of contact and friction to question how we would, with limited
resources care for this technical infrastructural rot, and try to
maintain the collective practices that go beyond them.

In questioning the horizon of these sedimented norms and the
inflexibility we had felt both in the inherited configurations of
technical docs as well as the social relations of how they were shared
in practice, we wanted to make room to care for these frictions. To do
this we coalesced these practices towards In-grid's own background and
promise of collective access. The sub working group of In-grid members
met many times together over on greenhost jitsi video calls as we often
do to have collective dialogue. Together here we reflected on how we
wanted to enact these practices, thinking about what structures and
formats made room for this information to be understood and configured
otherwise by us.

By making wiggle room within these constrained practices, we made room
for them to be flexible towards our ways of working and politics. Here
we were also keen to inquire how we could form wiggle room for others to
disorient and configure out these docs otherwise for them selves. In
this move we wanted to question the expertise and knowledge within these
systems, moving it from the isolated and determinate super users to
chaotic collective in practice. In this motion bring a plurality of
bodies into contact with these infrastructure to make sense of them from
many points of impact, friction and knowing. This approach aiming to not
just disorient the compliant technical expert and designer but to make
room for the figure of a disobedient user who brings with them their
community and context.

\hypertarget{in-docs}{%
\subparagraph[In-Docs]{\texorpdfstring{\protect\hypertarget{anchor}{}{}In-Docs}{In-Docs}}\label{in-docs}}

To question the barriers of technical language and sedimented formats of
technical docs I suggested we orient to crip studies and Kelsie Acton's
chapter \emph{Plain Language for Disability Culture} (2023). This
chapter by Acton poses how to approach language, structuring and
formating to re-configure them into more accessible plain language. This
is not meant as a a reduction of what is there but a reinterpretation of
how we represent, translate and figure these relations and dynamics
to/with others in accessible forms. It is also meant as a way to open up
different discourses, such as technical ones like these, to be within
reach of the disabled community. In this approach challenging the scope
of what disabled people (and others) can have expertise around, access
to and intimacy with. Here informed by her concept we actioned this
inquiry by firstly being conscious of Acton's principles\footnote{``Note
  on writing: This chapter is written in what I call a semi- plain
  language style. This means I do the following: - Use an active voice -
  Mostly use the 6000 most common words in the English language - Use
  short sentences - Use 14 point font - Use''I" and ``you'' " (Acton,
  2023)} but also thinking through how we can localise this dialogue
into technical docs and our own capacities. To do this I turned to
Acton's plain language version of Alice Wong's anthology of essays
called \emph{Disability Intimacy} (2024), which informed us how to
orient towards plain language. We situated these alongside the Web
Content Accessibility Guidelines (WCAG)\footnote{WACG Docs -
  https://www.w3.org/WAI/WCAG22/Understanding/} to think about how a
Plain -ish Language WCAG (P-ish-LWCAG) might be oriented.

With this in focus In-grid started to reconfigure the limits of docs by
trying to make room within their norms for docs to be improvisable and
knowable by different capacities and approaches. One way we did this was
by making more room in the introductory sections of docs to hold more
information about the background of these infrastructures, and
discussing what these technologies are and why we are using them in this
infrastructure. We also made sure to take care of and cover the
technical elements that the norms of docs from our experience would
brushed over or straighten out. With this metaphorical curb cut in the
introductions to sections, we aimed to form ways in for people to relate
to these practices who had divergent backgrounds to the ones where these
technical procedures had emerged from. In doing this we made room for
bodies which did not fit that of a normid super user or expert and
instead highlighted this misfitting, its frictions and points of contact
as generative sites for that body to re-imagine and reconfigure these
social and technical infrastructures.

Through this approach we also emerged our own syntax and formats for
clearly signing both links and technical code elements. For links this
oriented going beyond descriptive links to give extra short descriptions
below them for people to get a better idea of what lay beyond. For the
code inserts we started to form our own standards of how to make the
technical syntax docs encoded more legible and improvisable. To do this
in code exerts we highlighted \textless variables\textgreater{} with
these brackets, and followed this up with how we had run them ourselves.
There is an example quoted below. We did this to also emphasise the
readers capacity to improvise and adapt these practices to the persons
own local context. With this trying to make room to run them as
non-procedural practices which do not orient them to a specific user and
determined technical infrastructure. In this motion troubling the
procedural deterministic norms of docs to instead make room for people
to access these practices and knowledges through divergent disorienting
dialogues.

"Syntax:
\texttt{sudo\ tinc\ -n\ \textless{}NETNAME\textgreater{}\ init\ \textless{}NODENAME\textgreater{}}

So we did: \texttt{sudo\ tinc\ -n\ systerserver\ init\ servpub}"
(In-grid, 2025)

Towards the end of writing the docs we also found through extended
networks outside of Servpub of an earlier documentation from the XPUB
docs\footnote{https://pzwiki.wdka.nl/mediadesign/Tinc} as well as Lurk's
\emph{RUN YOUR OWN}\footnote{https://things.bleu255.com/runyourown/VPN\_with\_Tinc},
that the VPN parts of the Servpub infrastructure were based on. These
clearly showed how to setup these systems and would have saved us lots
of time configuring out these infrastructures. It would have been
sensible to follow the efficient ``Don't repeat yourself'' (DRY) mantra
of course, but in some ways this repetition, or what \emph{Get the frac
in} (Maier et al., 2020) might explain as a ``Tic'', lead us to repeat
the same stuff but maybe as plain-ish stuff. At this later stage we were
also facing the fact that the VPN elements of these docs (TINC) were
becoming deprecated. Feeling the impact of both of these frictions we
decided to reorient again from the norms of docs being an active site to
pose these docs as more of an archive documenting and make knowable this
infrastructure's background. Here forming wiggle room within our docs
for these histories and genealogies of infrastructures, and to share how
different bodies, relations and practice can transform and disorient
these norms when we bringing them in touch.

\hypertarget{in-practice}{%
\subparagraph[In-Practice]{\texorpdfstring{\protect\hypertarget{anchor}{}{}In-Practice}{In-Practice}}\label{in-practice}}

When reflecting on the frictions that emerged when we configured out
these infrastructures we contemplated how to make wiggle room for the
dialogues we had together. These processes and practices of making space
to reflect on what is being done and how, comes both from the emerging
internally methods of In-grid, but have also had an impression made on
them when being in contact with other collectives of Servpub
collaboration such as Creative Crowds and SysterServer. It also builds
up with Hamraie's notion of ``design friction''(2023), centring those at
the site of impact to re-imagine, redesign and reconfigure the
infrastructures and systems they are in touch with. Through these
approaches we dedicated time to caring for, feeling out and inquiring
into the figures we encountered together within the configurations of
these infrastructures. This for me personally was the likes of rubbing
up against Tinc's logo (fig.~\protect\hyperlink{fig:tincLogo}{1}), an
Apache attack helicopter, and being reminded of Isabel Waidner's
\emph{We are made of diamond stuff}, where the narrator questions the US
military naming practices (2019, pp.~18-23). This is where ``They use
words as weapons, they use weapons as weapons, and sometimes both come
together in the Boeing CH-47 Chinook''(ibid, pp.~22), but also in the
case of Tinc and it's Apache logo, how private networks are often
conflated with and depicted through militaristic violence and colonial
dominance. With a decolonial crip approach I would also understand with
Hamraie's crip technoscience\footnote{``A crip technoscience theoretical
  framework builds on feminist and decolonial technosciences to
  differentiate between technologies developed through models of
  disability- as- pathology (often in the name of rehabilitating injured
  soldiers) and those derived from dis-ability culture communities,
  where technology supports embodied differences and interdependent
  socialities.'' (Hamraie, 2023, p.~308)}, and actioning how we can
start to retreat these technical and social networks of privacy away
from these sedimented roots of militaristic and colonial violence
through community actions of access.

\includegraphics[width=3.38889in,height=1.20833in]{./media_05_In-Configure-Ability/Pictures/0.png}

Figure~1: Tinc logo which shows a grey scale Apache attach helicopter
behind tinc written in lower case black writing

In action and as a group we also found members being lost from graphviz
diagrams, maps and representations of this closed VPN network and
leaving them unable to help maintain or know these infrastructures
easily. This was mainly due to Tinc being both hard to access across
In-grid member's variety of operating systems (OS), as well as not much
time given by mentors to access Tinc from non-linux systems. As In-grid
we had to find hacks and improvisations to work together around these
limits, for example using Linux terminals on another OS. There was also
the for ever forgotten passwords for the for ever layers of security
that these zero trust network norms of domination entail, and which we
were not used to accessing. Working within Servpub we also encountered
more poetic, symbolic and figural naming practices of computational
systems which made wiggle room within them to re-orient these systems
througha background of feminist histories, language and variability.

These configurings out in practice are often impossible to portray
through not only a static set of technical docs, but also essays like
this here. They are about knowing through practice, and by taking
actions within and around the materiality of these technical and social
infrastructural relations and to feel their impact on your body. Instead
of trying to cure these inevitable gap between theory and practise,
In-grid oriented towards developing workshops for people to practise
these docs and network practices through configure-able methods for
themselves. When an opportunity arose for us to share this work at
4S/EASST 2024 in Amsterdam, we went about thinking how to formulate
these practices into the first iteration of our workshop called
\emph{Practising Protocols}. This workshop accompanied a panel we
organised and brought together with other members of community groups
and collectives organising community infrastructures, some from the 8M
strike. This included Donna Halford-Lovell from NEoN Digitial presenting
their community work around digital access, Julia Nueno presenting her
work with app delivery driver communities forming counter apps to
unionise and empower with, Femke Snelting from TITiPI sharing their
Counter cloud imaginaries, tactics and stickers, as well as members of
SHAPE talking about their work.

Bringing the docs into contact with this plurality of community
politics, imaginaries, and practice, also provoked us further to make
wiggle room for these collective differences of context, needs and
sensitivity in these practising protocols workshops. The name of
practising protocols itself emerges from both its feminist STS roots,
but also through a crip understanding of protocols as a place to dispute
expert knowledged of systems through counter protocols\footnote{``The
  feminist STS concept of ``protocol'' (Murphy 2012) describes
  methodological practices that become both standardized and reiterated
  in pursuit of particular politi cal goals. Crip making adopts
  protocol, alongside expert knowledge, as a site of inquiry into design
  methodologies more generally." (Hamraie, 2023, p.~311) \#\#\#
  Configure-ability In-Workshops}. In doing this we thought of how to
trouble the centralised and procedural practices of infrastructures
through making more room within the practices they emerge from to
localise and situate this knowledge and expertise with that of the
working group.

For this initial workshop at 4S/EASST we decided to crop in to the docs
to a very basic process of getting into one of the Servpub autonomous
servers and collectively editing a html file that was being served
online. This HTML would then hold the collective protocols we made
during this workshop. We did this as this conference has attendance from
a mixture of backgrounds, but is predominantly not a technical expert
space. This setting offered us room to test what it meant to make these
technical docs and their practices configure-able to people of these
varying backgrounds, capacities and expriences. Within these steps of,
sshing into the server, entering TMUX\footnote{\href{https://github.com/tmux/tmux/wiki}{Tmux}
  is a terminal multiplexer for Unix-like operating systems. It means
  you can share or collaborate in one terminal session in multiple
  devices.}, editing a HTML doc in Nano\footnote{\href{https://www.nano-editor.org/}{Nano}
  is an accessible terminal base text editor.}, and exiting the admin
session together, we made room and took our time to question what we
came into contact with, felt friction around and wanted to configure out
otherwise together. This iteration and context gave us great feedback
for how to run these workshops, with the final outcomes published
publicly on the
\href{https://ci.servpub.net/in-grid/collective-infrastructures}{panel
blog} and within the \href{https://servpub.net/ci_protocols.html}{html
doc} we edited live together. From this initial iteration of the
workshop we felt like \emph{Practising Protocols} were a promising way
of making room for groups to configuring out infrastructures together
through an accessible process that made room for their embodied
sense-making and expertise. In this iteration and workshop session it
made space for us to emerge with the group a dialogue around how these
practices and processes were pre-configured and imagined through
specific logics of servitude, users and bodily relations, and ones where
we as a group could start to discuss and form our own paths and
orientation within, around and/or away from them.

This workshop and panel combined worked well to make room for accessibly
dialogues around the technical imaginaries and configuration practices
of network infrastructures and collective organising. It formed a
context where In-grid and participants could question how communities
and collectives come together to form their own approaches, relations
and figures locally. It was also a fairly radical to do this with a
panel of ``experts'', that was run by a trans*feminist collective, and
with participants from digital arts charities, union organising and
radical counter cloud imaginaries at 4S/EASST. This could also be said
for the accompanying workshop that made room for conference members to
not only access the terminal and technical network practices together,
but to disorient them into a social dialogue in which we configured out
these technical practices in action together!

It was also interesting to reflect on how radically different the points
of friction and reflection were in this workshop at 4S/EASST, compared
to the next iteration with In-grid. This was of course because it was
with different people in a different contexts and with slightly altered
content. Still though it also shows to me the generative capacity the
\emph{Practicing Protocols} workshops have for disorienting the norms of
network configurations by making room for divergent and situated
dialogues that bring technologies in touch with and reverberating from
the bodies they impact.

The workshop I am bringing into focus here emerged from the
aforementioned long tail of in-flex-able inquiries into
configure-ability in-docs. This workshop is a focus of this inquiry as
it is the point at which we made room for these practices, that were
made know-able through smaller working groups on the Servpub project
over long periods, to be brought back and made accessible and known to
the larger group of In-grid. The scope for this workshop's actions were
to do the initial configuring out and setting up of our collective
server and network infrastructure, and which has led onto making room
for In-grid members to build technical capacities and emerge future
collective and community projects from. The actions of this workshop
were contextually quite simple ones, and oriented setting up our first
virtual private server (VPS) hosted with Servus and in doing so
providing In-grid members with their ``user'' accounts. Along side this
technical practice we also figuring out of our own itteration of
protocols and intentions for this platform as we came into contact with
it and made sense of these frictions. This simplicity of the workshop
also aimed to accessibly welcomed the wider group of In-grid members to
this server space, and through this bring these network practices
in-touch with In-grids background to make room for us to configure out
how we want to disorient this table together. Playfully I named this
workshop \emph{(¬}¬)/' Serv(er)ing In-grid\_ as we were not only setting
up In-grid with a server, but we were also configuring out together what
In-grid would serve there. Following on from the methodology of this
thesis and the emergence of these workshops, we began this inquiry into
how we wanted to imagine, tell tales of, and configure out In-grids
in-frastructure together.

Fortunately I had managed to get a bit of funding from the University
for these workshops, which enabled me to go up to London and do the
event in person with In-grid members. Even though this was the case a
number of In-grid members were running an exhibition in east London at
the time, and so Yewen Jin who had volunteered to look after the
exhibition asked if we could make it a hybrid event, so we pivoted onto
a video call. In this workshop funding I had also factored in
refreshments into this budget which meant we could eat and commune
together before/during/after the workshop. This was important to bring
these social, caring and joyous elements into our in-frastructure. For
the access needs form I also made room within its inflexible limits to
be oriented through pleasure by not only providing inputs for people's
access needs and dietaries, but also for communal joy and indeterminate.
I did this by asking what people's favourite snacks and treats were as
well as for them to suggest music for the workshop playlist within this
access form.

Sunni Liao had kindly volunteered their flat for us to meet and do this
workshop in. That morning before the workshop me and Sunni had met up
and gone climbing like we do together when I am in London. We caught up
about life whilst we warmed up and fell off the wall many times. We
rushed back to their flat to get ready for the workshop, and prep
everything. Sunni cooked up some tasty food for us, while I got the
treats and workshop ready. Other Members of In-grid started to arrive
and we caught up and got ready together. Running slightly behind we ate
as we began the workshop.

With Yewen online and the rest of us settling in we started to go
through the workshop and practised our protocols as we configured out
our server for the first time. Most members of In-grid, especially the
ones who were present, are fairly computer and coding literate, with
some having existing knowledge of server practices. This meant for the
workshop, I could put more focus on feeling out the protocols we were
actioning and making room within them to be configure-able by our sense
making of relations, approaches and embodied expertise. Within the
slides I edited from the Servpub docs I discussed earlier, I made wiggle
room for us to reflect within each step of this processes, focusing on
these points of contact and inquiring into how In-grid would
collectively figure out this configuration locally. These spaces of
reflection are named in context of their actions and were made up of
\emph{Coding protocols}, \emph{SSH protocols}, \emph{Sudo protocols},
\emph{User protocols}, \emph{Digital safe space protocols},
\emph{Collective working protocols}.

After this collective figuring out I made space for us to write
In-grid's first collective server manifesto together. In doing this I
made room for us to reflect on the feminist server manifesto (Constant,
2015) and form our own alignments and orientations to/from it. From here
we itterated from it our own manifesto as a way to coalese these
knowledges from practices of protocols and collective sense-making into
this manifestos background. Here manifesting a server as we manifest its
manifesto. The resulting Femfester manifesto with a small intro was then
published in first edition of \emph{Artists Running Datacentres} (Simms
et al., 2024). Below I reflect on this collective inquiry and the
\href{https://femfester.in-grid.io/}{notes} from the workshop to give
insight into what we came into contact with and how felt and made
friction when configured this collective infrastructure out otherwise.

\hypertarget{coding-protocols}{%
\paragraph[Coding
protocols]{\texorpdfstring{\protect\hypertarget{anchor}{}{}Coding
protocols}{Coding protocols}}\label{coding-protocols}}

This room was made at the beginning of the workshop and before we
started the technical parts of workshop to make space for us to reflect
on and propose our codes of conduct together before we encode our social
and technical infrastructures through them. We questioned how we desired
to work together on this server and what relations we imagined and
planned for ourselves. Here we focused on interdependence, on doing
things in pairs, keeping people in-sync and enabling a-sync processes.
We also focused on working slowly, being soft, and double, no triple
checking before we do things. There was also a focus on documentation,
not only as a way of keeping people in-sync, but also as a way to roll
back or undo decisions and actions (where possible).

\hypertarget{ssh-protocols}{%
\paragraph[SSH
protocols]{\texorpdfstring{\protect\hypertarget{anchor}{}{}SSH
protocols}{SSH protocols}}\label{ssh-protocols}}

Here we formed rubbed up against and wiggled room within the secure
shell (SSH) protocol, which is the protocol that lets you log into the
server. This is the first point that we came into contact with the
server, and SSH configures the way we can connect, interface and be
intimate with our server. There is little written feedback in the
workshop notes on the pad, apart from ``1. When creating metaphors
mention what is being replaced''. This note leads from a conversation we
had about the embedded metaphor and figure of SSH as a hand shake.
Within this conversation we opened up dialogues around the figures we
want to replace. Instead of a handshake which reinforces able bodiedness
within a normid user and with that the hierarchical relation and
inaccessible contract making between particular kinds of bodies. Here we
started to imagine what metaphors we wanted to configure our relations
to our infrastructure through. This particular friction did not generate
any outcomes of counter figures, unlike others latter on, as I feel we
may have still been warming into the process. It did though open up this
dialogue of how we wanted to move towards other figures of our own. We
wanted to find a practice and relation where we do not negate or hide
these histories in our retreat from them, but re-orient and reform their
bodies and relations through the impression of our retreat.

\hypertarget{sudo-protocols}{%
\paragraph[Sudo
protocols]{\texorpdfstring{\protect\hypertarget{anchor}{}{}Sudo
protocols}{Sudo protocols}}\label{sudo-protocols}}

Once logged in to the server through SSH we reached \emph{Sudo}, which
is the command ``super user do''. This is needed to do most of the admin
tasks on a server as it gives you the rights to do so. Here we found
friction with this sort of responsibility and expertise being
individualised to specific users and their bodies. In response to this
we turned to orient around interdependence as a way to slow each other
down, see things from a number of perspectives and hold each other
response-able. We also questioned how we could do this, asking questions
like ``can we track who is doing what as sudo?'' Instead in conversation
In-grid turned to questions of how we as a group could not formalise
trust through security, tracking, etc., but through social technologies
of interdependence, moving to further build trust between us socially,
instead of in solely through our systems technically.

\hypertarget{user-protocols}{%
\paragraph[User
Protocols]{\texorpdfstring{\protect\hypertarget{anchor}{}{}User
Protocols}{User Protocols}}\label{user-protocols}}

Once in sudo we then started to make everyone a user account for the
server. These user accounts are the ones that each of us uses to access
the server through and as a configuration orients the singular
individualised user on a device. Here we was a very generative space for
forming our own counter figures and found much joy in configuring
ourselves, our roles and how we wanted to relate to each other on this
server. In this reflective space we opened up a dialogue of how we
desired to reorient the norms and sedimented relations that the figure
of the \emph{User} represents. Instead of users, which fit a particular
normid body (Hamraie, 2017), and servers, which emerge from traditions
of hidding labour (Krajewski \& Iurascu, 2018), collectively we started
to think about how we wanted to reorient our relation to this space. To
figure this out we spent time chatting and writing up these imagined
roles together. This configure-able room for roles emerged In-grid as
maintainers, carers and caretakers of this space, but also personas,
players and characters. These all also oriented towards this server and
our infrastructure being a collaborator in our collective, moving it
from a normalised user/using hidden servers relationship, to one
oriented towards mutual care. This is highlighted within the middle of
the notes in the comment ``it is nice to be individuals in a
collective''. This is where we reflected on the bounds of these
individual users and the server they are situated on merge, but also
where the individual or user inside/outside of and around a collective
or infrastructure can offer us other imaginaries of how our relations
might be configure-able in that contact.

\hypertarget{digital-safe-space-protocols}{%
\paragraph[Digital safe space
protocols]{\texorpdfstring{\protect\hypertarget{anchor}{}{}Digital safe
space
protocols}{Digital safe space protocols}}\label{digital-safe-space-protocols}}

This wriggle room for reflection was made when members uploaded their
SSH keys to their user (carer/maintainer/character/collaborator/persona)
account so that we could turned off password access. This meant that
only a person on a device with that key and its password can access that
user on that server, making it ``secure''. This is the standard or
``norm'' of security for a server with a public IP like this, and
realistically does keep this space ``safe'' within the materialised
military imaginaries of a zero trust encryted web that we exist in today
(Spencer \& Pizio, 2024). This relation became a place we found friction
within these infrastructural inflexibilities that left In-grid asking
what other imaginaries of safety and privacy could we imprint on our
server from our background. Here I think influenced by the feminist
server manifesto we talked a lot about transparency. We discussed how
transparency even though performs the making visible of dynamics, is in
many ways a barrier in place that is there to make a system ``secure''
from intervention. In this performance of transparency is also the
filtering and orienting of these material practices and their background
into the abstract languages and representative figures of experts.
Instead we started to ask how we could make this relation more
configure-able through and understanding of critical access as an
alternative approach to transparency. This formed specifically of the
desire to not ``ring-fence around particular roles'' or their
knowledges, and the need to not have ``knowledge lie with one person''.
With these statements refusing this figure of the expert user, and
orienting to community and collective knowledges. To do this we thought
about how safety for us is not about static barriers of security or
hierarchical roles that reinforce each other, but to be intimate who we
are with and what we are doing. From this discussion our safety comes
then not from a checklist, charter or generalised deterministic set of
conditions, but from taking time to build trust and making room for each
other to form dialogues and exchanges in-flex-able ways. Within network
practices this was particularly relevant, as we did not want each other
to be intimidated (by the already intimidating technology), so forming
relations of trust that centred access of these technologies was key. We
came up with many protocols and practices for achieving this, from again
working in pairs/groups, researching/checking/quadruple-checking before
we took actions, and collectively demystifying and becoming experts on
what these protocols and technologies are to us.

\hypertarget{collective-working-protocols.}{%
\paragraph[Collective working
protocols.]{\texorpdfstring{\protect\hypertarget{anchor}{}{}Collective
working
protocols.}{Collective working protocols.}}\label{collective-working-protocols.}}

This point we joined up together within TMUX, which is a terminal
multiplexer. This sounds very complicated, but basically lets you work
collaboratively within one terminal together. I wanted to bring this
collaborative interface into focus as it does have many restraints and
frictions for collaborative working. These range from basic things like
classic obscure terminal copy and paste commands, but also that there is
only one person able to write at one time, and the possibility of
accidentally loosing sessions by typing exit (even in another command).
In the pad and session though the reflections focus mainly on technical
elements of TMUX as the limits above gave us a few technical issues
members couldn't quite get beyond. We were also running a little out of
time so decided to move on from this and to focus on writing the
manifesto together. Reflecting on this brief contact though, it was
interesting to see how collaboration in these technologies is configured
through very restrictive relations, and when a group like us, who
collaborate in a number of dynamics and approaches, found this
pre-configured relation too tight and misfitting to our needs. In-grid
was definitely not the imagined expert users of TMUX, but being in touch
with these constrained collaborative environments showed us how much
these relations needed us to form wiggle room within them, as they were
frictious to use.

\hypertarget{manifesting-femfester}{%
\paragraph[Manifesting
Femfester]{\texorpdfstring{\protect\hypertarget{anchor}{}{}Manifesting
Femfester}{Manifesting Femfester}}\label{manifesting-femfester}}

Emerging out of this foreground and the workshop's wiggling of room in
these technical protocols we turned our attention to the collective
writing of our initial Manifesto for this server. To do this we talked
as a group about how we wanted to orient our approach to this. To start
this off In-grid members chatted about what the \emph{Feminist Server
Manifesto} (Constant, 2015) meant to them, and what they thought of it,
with members being excited for us to take on iterating our own version
as it had been an inspiration to many in the group. We decided that a
collective reading of the original \emph{Feminist Server Manifesto}
would be a nice way to initiate this, and then follow it up by
commenting on, iterating through and configuring out our own manifesting
of it. In this reading we took turns to read lines in non-hierarchical
and improvised form with members dynamically taking up where others left
off, and echoing back to earlier collective reading practices and
performances we have done\footnote{https://www.in-grid.io/projects/residency-event-maggie-roberts/}.

We followed this up by doing another In-grid practice of synchronous
free writing in a shared pad. This is a practice we have emerged that
enables members to anonymously let ideas and thoughts out, making it a
very relaxed way to start ideation and get a feel of what the group is
making sense of. After this free writing we started to form alignments
with the manifesto by writing out the original lines, talking through
them and figuring out how ``our lines'' oriented from theirs. We only
had time to go through half of them in the session, as people had time
constraints, but here again we were in-flex-able. To do this we turned
to an a-sync work flow working on the etherpad and online. Through this
process of reorienting the manifesto and feeling out a few points where
we flexed from their path. Here we re-oriented many of these issues
through critiques and figures of access, asking how we as a group can
imagine and materialise these approaches in our network practices. This
ranged from caring for the frictions of accessing a server collectively,
to making those difficulties visible and accessible to others. In this
orientation we formed our own impact centred critical access informed
dialogues and practices that disoriented ``transparency''. In-grid's
central figure of reorientation of this emerged from our dedication to
interdependence as a pivotal to our manifesting. The original
\emph{Feminist Server manifesto} does clearly come from community and a
place of interdependence, but we wanted to emphasise interdependence as
the figure that made room for In-grid's manifesting. In this move we
also reoriented from not apologising for not being available (all the
time), to learning to communicate our capacities and feel out and take
care around our limits, both individual/collective. This for me also
opens this manifesting to being a practice that changes over time, and
one that aims to not only reveal but also build up our capacities and
abilities within relation. We did in many places not wiggle from the
original manifestos path though, finding alignment in the refusal of
security, of exposing insecurities and avoiding ease-of-use. Again
though we found it important to reaffirm these moves through
abolitionist concepts and wordings. Over the next week or two we worked
together asynchronously to edit and finesse together this manifesto,
along with a short intro, for publication with Servus and the first
edition of their \emph{Artists running datacentres journal} (Davide
Bevilacqua, 2024). Here it was engulfed in a further plurality of
research, interviews and practitioner knowledges from across Servus'
community.

This manifesto aligns with the provocation of the original Feminist
sever manifesto (ibid), as well as with this researches approach of
flawed and crazed crip manifesting (Kafer, 2017, 2023). This is where in
its essence the Femfester also yearns to be iterated through its
misfitting in action, and frictions made and felt within complex systems
and configurations of community infrastructures. This is our iteration
and attempt to wriggle room in the infrastructural bounds we have rubbed
up against, and with this manifesto we tried to imagine the room,
practices and capacities for flexibility around In-grid's
collective/individual bodies' needs to be present within our futures.

\hypertarget{in-reflection}{%
\subsubsection[In-reflection]{\texorpdfstring{\protect\hypertarget{anchor}{}{}In-reflection}{In-reflection}}\label{in-reflection}}

In reflecting on our collective processes, our \emph{Serv(er)ing
In-grid} by \emph{Practicing Protocols} to \emph{Femfester} together our
first server manifesto I aimed to offer up how configure-able methods
can make room within relations for the flexibility our collective and
communal bodies need within our sociotechnical infrastructures. This set
of inquiries made room for In-grid (and I) to take configure-able
methods up and to not only practice but also process our manifestation
of our collective infrastructures. Reflecting more closely on the
outcomes of the workshop, I think it worked well as a method to wiggle
room within these technical practices through critical access informed
disobedient interventions. In-grid here taking our sweet time to form
collective imaginaries through these direct actions. With these actions
making room for our embodied situated knowledges as a group to orient
how we want to configure these social and technical reltions of network
infrastructure and collective organising. For me these inquiries into
in-configure-ability have emerged as a place for In-grid to challenge
the norms and sedimented relations of network infrastructures through
making room for a wider relational and political orientation which
retreats from the norms of automated security politics. To do this we
instead turned to social practices of forming community trust,
interdependence and care. Security politics were primarily troubled by
us through notions of access as an axis that can feel out bounds and
enable flex-ability within hard systems to build dialogues of trust and
practices of interdependence. With this movement we aimed to move beyond
the myths of transparency by interdependently building up our collective
capacities of access, and to try to centre us all as experts of our
collective systems, politics and limits.

\hypertarget{old}{%
\subsection[old]{\texorpdfstring{\protect\hypertarget{anchor}{}{}old}{old}}\label{old}}

In reflection though, and clearly through the accounts I gave, I could
have given better time allocation for these exercises, but it is also
good to see the ways we can find flexibility in our interdependence to
pivot our work within our limits. In recollection this may have been
better split between two sessions, or longer over one day, but within
the time constraints and limitation, it worked well to demonstrate the
methods and process. In a similar vein I think I would have liked to
have re-ordered the session, sandwiching the technical (con)figuring out
practice in-between the manifesto(ing). I felt in this workshop the two
elements, the material (con)figuring out and the collective
manifestoing, were somewhat separate, and finding a approach to entangle
them more would have been interesting.

\hypertarget{in-configure-ability-1}{%
\section[In-configure-ability]{\texorpdfstring{\protect\hypertarget{anchor}{}{}In-configure-ability}{In-configure-ability}}\label{in-configure-ability-1}}

In this Chapter I reflect on my inquiry into the intersectional work I
undertook as a member of In-grid. This inquiry builds out of In-grid's
slowly sedimented collective practices and aims to highlight how through
first person action research and disobedient action research methods I
have introduced critical access as a critical framework and practice to
reorient our ways of relating to and coalescing tables of social and
technical infrastructure. It specifically orients how we accessed and
were accessed by practices and their knowledges, reflecting on how
practices and their knowledges have been oriented to be accessible to
the wider In-grid collective, 8M communities and wider public.

To do this I am following the prior chapters pre-figuring of
configure-able to go beyond that ``definition'' and instead focus on it
in action and through reflective inquiry. I do this firstly by setting a
background for these figures to emerge from, acting as a ground for them
to be contoured and figured out from. I then inquire and reflect on
moments and relations of misfitting and inflex. These moments of inflex
and misfitting of bodies enabled us to feel out where there were points
of friction and constraint that revealed seams for us to wiggle into.
Inflex as a word here also highlights the making crooked of something.
Through both crip studies and queer theory, with Kafer and Ahmed, we can
understand this as people with disabilities being othered, or made into
crooked crooks for society to lean on and oppress to differentiate a
normal or superior from. With Ahmed and this research's framing we can
query this-holding-things-in-place-of-inflex, and ask instead what
things might be able of when configured with flex-ability. Doing this I
hope to enable an inquiring into what contingencies we might be able of
when we reorient our approach to not be so dominant. In this work these
points of inflex are the contours that outline these specific localised
figures. This framing of inflex is also meant to move these dialogues
beyond the problem/solution axis of curative design practices, and
instead focus on how we address and orient towards not only validating
feeling out these frictions, but enable that feeling and figuring out as
part of their process, forming varying positions, orientations and
approaches that a crip intersectionality invokes.

Working through this framing, I reflect on In-grids collective inquiry,
as well as my first person inquiry, into how to address these
inlfex-abilities within infrastructural coalition with
in-flex-abilities. This is a recurring word play from us, where we get
in- to all sorts of things (and trouble). Here though focusing on how we
built up our own abilities to flex and practice access in contexts and
relations of technical communication and infrastructuring, as well as
collective organisation, decision making and feedback. These are again
not answers, cures or generalised solutions but reflections on how we as
a group have felt frictions from inflex, and in response enabled
in-flex-abilities to make wiggle room in these seems and through
configure-able practices of access.
