\hypertarget{disability-justice-and-life-affirmation-flipping-the-table}{%
\section[Disability justice and life affirmation flipping the
table]{\texorpdfstring{\protect\hypertarget{anchor}{}{}Disability
justice and life affirmation flipping the
table}{Disability justice and life affirmation flipping the table}}\label{disability-justice-and-life-affirmation-flipping-the-table}}

Now is the time for me to try and approach a crip table. This chapter
makes present how this work approaches this crip table and axis and
inquire into analyse the institutional tables I am engaged in as a
disabled practitioner and researcher. It gives some depth to why these
tables have been impossible for me to approach as a crip, let alone get
a grip on. This isn't because the physical institutional setup have
barriers that stops me from accessing them, nor does my chronic health
often debilitate me, and fortunately my care system somewhat validates
my disability\footnote{As some disabilities are invalidated, or
  indeterminate within medicine, often leading to some of the most
  exhausting and painful experiences, that are still somehow
  invalidated.}. Nor is it that I am incapable of doing ``quality''
research. It is because of the political/relations of these institutions
that burn me (out) every time I am in touch with them. It is because
when I come to these tables they cut away at my capacities. Here, as I
will go on to discuss further, I am thinking with Ahmed's notation of
sedimentation and the determined orientations of work that result in
repetitive strain injury (RSI) from bodies being held in place, and
which reinforce work to break them before they start.

When I come to these institutional tables I am neither willing nor able
to follow these exhaustive straight paths they set, as with my fatigue I
will need a nap on route, maybe make a stew, go for a walk to digest and
take a bath to replenish and care for my bodymind. By the time I am
ready and able, the path has often grown over, or I might have forgotten
exactly what I was doing or where I am (going)\footnote{No matter who I
  am.}. These crip lines are deviating orientations of survival and of
life affirmation. What does it mean to be able to lay with these very
queer and crip paths? Where can we hold this crip stuff and make times
that celebrate them as places of generative access in conversation,
instead of invalidating, problematising and backgrounding them from the
tables where they have always been present but ``hidden from view''?

Through discussing these backgrounds this chapter acts as the
positioning statement of the research. It aims to contextualise the
research within queer and crip theory, specifically Sara Ahmed's work
(2006) on Queer Phenomenology; Alison Kafer's work (2013; 2017; 2021;
2023) on crip time, her political/relational model of disability, and
her crazed crip manifestings. I also orient this towards my focus of
technoscience with Aimi Hamraie and Kelly Fritsch's Manifesto (2019),
Hamraie's wider work of access friction and access knowledge. This
framework is used throughout the thesis to not only frame the analysis
but also situate the work from my experiences of institutional relations
to access, computing and care as a crip \& trans teacher, coder,
collective instigator and researcher. Following Sara Ahmed I aim to
offer an approach to how ``bodies take shape through tending toward
objects that are reachable'' (2006), specifically how this shaping takes
place when approaching accessible futures for crip and disabled folks at
the institutional tables of research, computing and collective practice.
Building from Ahmed's figure and framework of the table as an
``orientation device'' this chapter asks what are these tables I am
approaching, and why do I (often) retreat instead of taking a seat? It
makes room to question what the normative and inherited
political/relations of these institutional tables are, and how crip as
method and situated experience, practice and knowledges makes room for
us to imagine, coalesce, disorient and manifest other collective tables
into being.

This framework held in parallel with the
\href{../../02_Crip-Tic_of_Vignettes/02_Crip-Tic_of_Vignettes.md}{02\_Crip-Tic\_of\_Vignettes}
chapter, forms an orientation for this research that proposes crip
manifestings, manifestos, protocols and infrastructures as a way to
re-orient the norms of computational practices through a cripping
disorientation. This disorientation asking us to refuse the pull, the
heckles and obligations of tech practices and to listen to our own crip
and queer axes, ones of life affirmation and radical relational
politics, to form our own deviating lines that fit our collective and
community bodily horizons instead of cutting away at them. This thesis
draws other lines of survival, of necessity and of crip movement making,
where we refuse orientations of tables and relations that aim to break
bodies before they begin. Instead, it seeks to form practices to
manifest together orientations and movements that can turn these tables
towards stories of collective liberation and care.

\hypertarget{orienting-the-table}{%
\subsection[Orienting the
table]{\texorpdfstring{\protect\hypertarget{anchor}{}{}Orienting the
table}{Orienting the table}}\label{orienting-the-table}}

In Ahmed's Queer Phenenology, she keenly defines orientation\footnote{``Adding
  `\,'orientation'\,' to the picture gives a new dimension to the
  critique of the distinction between absolute space and relative space,
  also described as the distinction between location and position.''
  (Ahmed 2006, 12)} as a way to situate the dialogue of direction. This
move changes the scoping from a generalised and centralised dictation of
direction that wider modernist philosophy denotes, to a situated and
relational navigation of bodies. She does this by queering Husserl's
figure of a table, his orientation towards it and his knowing of what is
behind him as he does so (Ibid, 29). To do this she positions his back,
and backgrounding of things, as the point of deviation that a queer
phenomenology takes up in its analysis. At this inversion of direction
and position sits the ability to wonder what is behind, to meander and
imagine otherwise. Ahmed calls this ``turning the tables''\footnote{``In
  a way, a queer phenomenology is involved in the project of `\,'turning
  the tables'\,' on phenomenology by turning toward other kinds of
  tables. Turning the tables would also allow us to return, a loving
  return we might even say, to the objects that already appear within
  phenomenology, such as Husserl's table, now so worn. Such tables, when
  turned, would come to life as something to think `\,'with'\,' as well
  as `\,'on.'\,'' (Ahmed 2006, 63)}, where we situate a table by letting
its background sink in and by caring for the work and maintenance that
is often `\,'hidden from view"(Ibid, 63) to make room for these queer,
crip, social and domestic tables to be generative discourses within
research and knowledge creation. This is where tables become many, where
they deviate from straight standardised and projected 'known' formulas
and become complex assemblages, with histories, sedimentations and
horizons.

The ``bodily horizon''(Ibid, 55) is key to understanding Ahmed's figure,
as it asks, what do we orient ourselves towards? What do we bring
towards ourselves from our horizon, and what objects do we let slip from
view and out of reach? When we acknowledge we have limits to our
perception, that we cannot see the whole environment, image or figure,
we must be in a context. Ahmed's horizon queers phenomenology's horizon
and encourages us to consider our own orientation, situate ourselves
within our bodily horizon of social and domestic relations and ask which
orientations of space make our bodies feel out of place. With Nirmal
Puwar's `Space Invaders', Ahmed shows how non-white bodies in places
oriented for white bodies makes them feel ``out of place'' (Ibid, 133).
She says that these bodies are disoriented by the society and people
around them: ``people blink and then look again'' (Ibid, 135). For her
this disorientation is a queer slanting of space, disorientations of
bodies and that of bodies into objects. Ahmed poses
disorientation\footnote{``Disorientation involves failed orientations:
  bodies inhabit spaces that do not extend their shape, or use objects
  that do not extend their reach. At this moment of failure, such
  objects `\,'point'\,' somewhere else or they make what is `\,'here'\,'
  become strange.''(Ahmed 2006, 160)} as the point of failure, in which
a body is placed into a form with which they do not fit, where they are
objectified, and their needs are held out of reach.

When we ``feel out of place'' and disorientation is the norm, Ahmed
poses the movement of being in retreat, of moving away from, whilst
facing. In this retreat, she says, there is queer joy as we look back at
a past fading, into a future which can make room for ``the condition of
possibility for another way of dwelling in the world'' (Ibid, 178). In
feeling out of place at, disoriented by and in retreat from
institutional tables this research(er) asks what it means to move within
reach of another way of dwelling at these tables of computation,
collaboration and institution, and specifically one that cares for and
affirms crip and queer lives, experiences and knowledges.

\hypertarget{sedimentation}{%
\subsubsection[Sedimentation]{\texorpdfstring{\protect\hypertarget{anchor}{}{}Sedimentation}{Sedimentation}}\label{sedimentation}}

Ahmed also extends the bodily horizon to time, thinking of how
repetitions, sedimentations and tendencies shape bodies. She makes an
analogy here with a lump on her finger from where she grips her pen,
where her finger has built up and sedimented its form in relation to the
pen. Ahmed sums it up simply as ``what we `\,'do do'\,' affects what we
`\,'can do.'\,'\,'' (Ibid, 59). She refuses the hegemonic causality of a
focus on doing as a singular straight direction. Instead, she suggests
orientation is an expanding of a certain direction, that may
relationally restrict others, but is not determinative. Her definition
makes it possible for orientations and their sedimentations to not be
singular in focus, and to make room form them to be navigated and
approached within complex relations on our horizons and to take queer
paths between them. When this approach arrives at the regressive
mono-logics of institutional efficiencies, it disobediently opens up the
many other paths and directions that are contingent.

Ahmed also discusses RSI as a potential effect of inheritance,
sedimentation, time, repetition, and the determined table. RSI is the
strain and pain caused from bodies repetitively conforming to a tables
that does not care for them and figures them to be disabled before they
begin. Here, bent over, hunched and aching as I work, in this cramped
spot, is the perfect time to stretch out into Ahmed's queering of
orientation. With RSI, I overtly feel how my body wants to be
orientated\footnote{Mine is often in bed.}, but I am contorted by
inflexible hard systems, infrastructures and relations, through an
institutional table that can only orientate my towards an ideology of
efficient work. Through a queer orientation, Ahmed prompts me to listen
to my bodies, my aches and my axes, to feel what I tend towards or need
to retreat from.

It is helpful to question these relations of RSI through Ahmed's concept
of inheritance\footnote{``Such an inheritance can be rethought in terms
  of orientations: we inherit the reach ability of some objects, those
  that are `\,'given'\,' to us or at least are made available to us
  within the family home.'' (Ahmed 2006, 126)}, thinking through not
only where these projections of contorted bodies and relations have come
from, but also what they perpetuate. What are the horizons of these
inherited relations, and what is reachable from these positions? When
the straight/atomic home is orientated so that there is a ``requirement
that we bring home the''other sex" \ldots{} '\,'same race"" (Ibid, 127),
then it is orientated to be self-perpetuating, autopoetic, and demanding
a ``good likeness''. This orientation of a home demands that everyone be
``in line'' and maintain the norms and order of things, making sure that
the relations and hierarchies ``line up'' between generations. Ahmed
gives an example of drawing on tracing paper, copying lines over, and
the sophisticated apparatus that philosophers, researchers, designers
and institutional tables maintain to keep lines straight by ``holding''
things in place. These institutional apparatuses often smooth over any
lumps, bumps, pains and aches that could queer the line and make it
deviate. With Ahmed's approach I am empowered in some ways when I am
disoriented by these enforced, inherited and sedimented lines, as this
disorientation is a generative position of sense making, and to
re-orient towards other tables and practices to sediment new queer and
crip lines of affinity.

\hypertarget{queering-the-axis}{%
\subsubsection[Queering the
axis]{\texorpdfstring{\protect\hypertarget{anchor}{}{}Queering the
axis}{Queering the axis}}\label{queering-the-axis}}

To queer this sedimented straightening axis, Ahmed starts by tracing the
etymology of ``queer''. Queer in a spatial context means a twist in
space, and when we combine this with definitions of sexual orientation,
it poses a bent, deviant or crooked sexuality. Ahmed starts to queer the
straight line to make wiggle room to thinking about how to approach
relations on this horizon. If we are to move straight towards an object,
as a singular other from determinate positions, then we can only take a
straight path towards it\footnote{``The straight line would be that
  which moves without any deviation toward the''point" of heterosexual
  union or sexual coupling: any acts that postpone the heterosexual
  union are perverse, which thus includes heterosexual practices that
  are not ``aimed'' toward penetration of the vagina by the penis"(Ahmed
  2006, 78)}. She demonstrates this at the dining table where one man
sits opposite one woman. This emerges into the family table where we
have set sides and positions. Ahmed shows how this line straightens out
a queer or crooked one by pushing us to choose one side of the table.
Ahmed questions how the queer line is portrayed as oblique when held
against the straight line, even posing that the straight line needs the
queer one against which to appear comparatively straight. Ahmed
reorients this idea through her figure of the ``contingent lesbian''
(Ibid, 92), which queers Freud's notion of a ``contingent introvert'',
who positions queerness as a lack of finding a partner of the other sex,
and so turns to her own by default. Ahmed inverts this relation to
position straightness as contingent, and sees lesbians not as defaulting
to queerness in the lack of a suitable person of the other sex, but
defaulting to straightness and the other sex when the familial line and
sedimented straight norms put queer bodies out of reach. Here she is
thinking about this not as a ``coming out'' as queer or lesbian but as
``coming to'' queerness through desire. This reorients these relations
towards queer desire and away from the ``requirement'' of a straight
line. This makes room for me to understand how I approach the world
along my own crooked or deviant lines that take me towards different
people, communities, relations and in doing so shape my body and
sediment my own paths.

Key to this critique is a queer reading of Merleau-Ponty's ``sensitive
body''\footnote{"\emph{At the same time that we acknowledge this risk of
  universalism, we could queer Merleau-Ponty's }`\emph{'sensitive
  body,}'\emph{' or even suggest that such a body is already queer in
  its sensitivity }`\emph{'to all the rest.}'\emph{' Merleau-Ponty's
  model of sexuality as a} \emph{form of bodily projection might help
  show how orientations }`\emph{'exceed}'\emph{' the objects they are
  directed toward, becoming ways of inhabiting and coexisting in the
  world."} (Ahmed 2006, 67)} where Ahmed reflects on how inhabiting
space produces a space to be inhabited. For example, a soft bottom
creates a softer seat, that creates a softer bottom and softer seat.
This concept is used to figure how a queer twist in the straight line of
inheritance is itself conducive to forming a line that has room and
capacity to be twisted. To think this through, she takes up the
butch-femme dynamic of lesbian relations, and critiques not only the
hetero perception of needing sexual difference or an opposite to attract
but also queer critiques like Joan Nestle's ``phony heterosexual
replicas'' that mimic this. With the help of Judith Butler she poses
that the real need here is to question the necessity for difference in
queer attraction, of the ``requirements'' of desire and relations. Ahmed
instead positions these roles as performances as ``erotic possibilities
that can generate new lines of desire only when they are just that:
possibilities rather than requirements'' (Ibid, 99). In doing this she
is opening up the possibility for bodies to not be determined by set
relations but to be things to be desired, pursued, subverted and felt
out. In this move she plays on lesbian fictions of Butch-Butch relations
within Leslie Feinberg's \emph{Stone Butch Blues} as well as Lee Lynch's
\emph{The Swashbuckler}, to show the comical navigation of desire of
self within lesbian culture. She turns here though to state that these
performed lines and bodily horizons of butch-femme are not illegitimate,
but cautions against drawing ``a dividing line'' that can in itself
``make other forms of sexual desire unlivable''(Ibid, 99). In this
research, I have been playing with these ideas around queer relations in
the context of wider intimacies, from how roles such as users and
experts are performed with digital network infrastructures, to the
emerging of indeterminate collaborative and collective desires.

Ahmed takes Merleau-Ponty's notion of the ``sensitive body'' to discuss
how this ``coming to'' queerness shapes and forms our bodies through
being in relation with and touching other queer bodies. She explains how
being oriented towards other women doesn't just shape our sexuality, but
also what other objects are reachable on our horizon. She shares how
queer and lesbian desires move us sideways, taking us to different
associations and connections that are often invisible to others. As I
follow my own queer path she encourages the refusal of this
straightening up, to not look back or reorient ourselves to/around
straight heckles, however offensive they are, and to instead orient
towards my own queer promise. This queering of the axis is to stay
disoriented, to orient into possibilities that straight lines could
never map, and to touch and hold our bodies in our own economies. A
queer axis is one where the horizontal and vertical lines of inheritance
don't always ``line up'', can be ``off line'', and challenge the
sedimentations that hold me in place, force me to one side and keep me
out of reach of me needs and desires.

Removing the ``holding'' things in place, in this research when I move
this metaphorical tracing paper to other tables and relations; the
grooves in these different tables inducing a queering that animates
these sedimented lines to be drawn otherwise. This means that if I take
a seat at institutional tables as a crip and queer\footnote{If we have
  the capacity to.}, even if it isn't orientated, approachable or stable
to me, I am mutually affecting it. It means that when I bring myself, my
thoughts and research to other queer tables, of coalition and of crip
care, I can be comfy in our shared disorientation at hegemonic tables.
The research I am doing gets slanted with my body to my other queerer
axes. When I come to institutional tables along these queer axes, I take
them off balance, grasping at them and nudging them to an orientation I
need. This is a very queer dance that begins to happen when I refuse to
be held in place, or hold other things out of place. I start to be able
to be flexible in how I orient myself, begin to be able to feel my needs
and to form tables and relations of affinity that feel mutual,
constitutive and affective.

\hypertarget{transdisciplinary-practice}{%
\subsubsection[Transdisciplinary
practice]{\texorpdfstring{\protect\hypertarget{anchor}{}{}Transdisciplinary
practice}{Transdisciplinary practice}}\label{transdisciplinary-practice}}

Moving from a queer axis this research sits at the intersections of a
number of disciplines, so I find it important to reflect on Ahmed's
thoughts around transdisciplinary practice. In queering the axis and the
paths I take, I can think through what it means to not always know the
path, to get lost and connect things that aren't together when they are
held in place by sedimented norms. For Ahmed, potential naivety around
the proper way of doing things is a place of production where queer
lines are drawn and other paths are made. This means that there is great
strength in holding on to these institutional tables, however much they
may try to throw you off, or move apart. By being there, and by making
wiggle room\footnote{``Sometimes that is what we struggle for: wiggle
  room; to have spaces to breathe. With breath, comes imagination. With
  breath, comes possibility. We might in spilling out of the rooms we
  have been assigned, in our struggle with an assignment, mess things
  up.'' wiggle room blog post}, wiggling tables together and apart, I
create the room I need to move towards the queer lines and modes of
orientation that I desire. This room to breathe offer me the capacity to
touch, feel out and imagine what my needs are and what paths may make
these possible. Where do I have to move to be in reach of these needs
and desires? Through transdisciplinarity I have the agency to imagine
and practice what combinations of lines, what muddling of methods and
disassociation of discipline makes the room for the change I need now.
Much like Ahmed's navigation of butch-femme dynamics,
transdisciplinarity is not meant to de-legitimise disciplinary fields or
sedimented relations, but is meant to approach them as performances of
knowledge and practices. Orienting disciplines in this way means that I
can start to make room for what was once ``unlivable'' in disciplines'
inherited terms. Again, this approach in its disobedience doesn't
outright discredit the disciplines it works with but asks how their
performed orientations may cross and transpose one another, what that
could put in reach, and how making disciplines touch in other ways
reshapes the bodily horizons of the disciplines themselves. This moves
the focus from boundaries to orientations and how I can trace situated
queer paths between disciplinary lines for myself. In this research, I
bring together and move across fields of queer feminism, crip theory,
critical access, disability justice, STS, media studies, information
studies, artistic/creative practice, and software studies, to ask how a
crip network and collective organisational practices can be made
live-able between the institutions of computing and academia that
systematically invalidate disabled queer lives.

\hypertarget{the-crip-table}{%
\subsection[The Crip
Table]{\texorpdfstring{\protect\hypertarget{anchor}{}{}The Crip
Table}{The Crip Table}}\label{the-crip-table}}

Here I turn to the crip table, and in doing so aim to extend Ahmed's
understanding of what is in reach with contemporary crip theory. Ahmed
is in many ways an amazing commentator on disability and anti-ableism,
but here I turn to Alison Kafer's table, and to her work with her
\emph{Feminist, Queer, Crip}. Kafer's formation of the
Political/Relational model of disability, and discussion on Crip Time as
a concept makes room for me to question through more focused crip lens
how ``bodies take shape through tending toward objects that are
reachable'' (Ahmed, 2006). In doing so I am thinking through how
politics and relations form and hold disability in place, and in turn
how through community, interdependence and collective access in action
can make room for disability's indeterminacy to manifest many other
futures together. I also inquire into Kafer's figuring of crip time not
only as a way to hold many indeterminate bodies, but also as a way to
understand and navigate the scoping of actions taken in this research. I
reflect on the many scales that collaborative labours take, especially
around manifesting collective infrastructures, from the infinite small
meetings and feelings that I could never capture, to scales of
maintenance that could have ended yesterday, or could stretch beyond my
lifetime. With this, I also slip onto Kafer's table the \emph{Crip
Technoscience Manifesto} (2019) by Aimi Hamraie and Kelly Fritsch, to
re-orient these dialogues towards social and technical infrastructures
and science technology studies (STS). Here I orient the crip as a social
and technical hacker, reforming and manifesting infrastructures through
direct action, affective feedback and refusal that centre crip knowledge
and experience within making and doing. With this I ask what does it
take for crips to make and sit cozily at tables of disability justice,
access and life affirmation within and around institutions and
infrastructures of research, computation and care? I want to approach
these questions with others, within our capacity, with passion, pleasure
and joy, asking: where/how can we make room to imagine and practise
other political relations of liberation and affirmation, now?!

\hypertarget{politicalrelational-model}{%
\subsubsection[Political/Relational
Model]{\texorpdfstring{\protect\hypertarget{anchor}{}{}Political/Relational
Model}{Political/Relational Model}}\label{politicalrelational-model}}

Alison Kafer's \emph{Feminist, Queer, Crip} is a foundational work for
understanding how disability can be analysed from a crip perspective and
approach. It looks to complicate the way that disability is represented
and mediated, moving on from both the medical model and social model of
disability, to one that can hold the diversity of intersections queer
and disabled bodies experience as well as making room to critique the
wider implications of a cure narrative. In the medical model of
disability, as Kafer notes, quoting Simi Linton, ``the proper approach
to disability is to''\,`treat' the condition and the person with the
condition rather than `treating' the social processes and policies that
constrict disabled people's lives."" (2013, 5). The medical model treats
the sickness or ``abnormality'' of disabled people on an individual
basis and in doing so ignores and invalidates the background to their
potentially disabling social relation. In doing this the medical model
also ignores the experiences of disabled people whose bodies are not
only most affected by these orientations, but also have the affective
and situated knowledges that are essential to their nuanced affirming
care. The medical model sees that ``solving the problem of disability,
then, means correcting, normalising, or eliminating the pathological
individual'' (Ibid, 5). A cure here can mean returning to being ``able
bodied'' but it can also be interpreted as society being cured of
disabled bodies. This medical model draws a line that forces disabled
people to one side, and off of the table systematically through
isolating them from it financially (Ibid, 39), as well as through
eugenic regimes of genetic selection (Ibid, 76) and curing ideologies of
technological enhancement (Ibid, 107).

A later and more accepted model within the disabled community is the
social model, which aims to draw attention to the social aspects and the
background of disability that the medical model invalidates. It does
this through an `impairment/disability' axis, which makes room to
understand what a person is capable of and how a society/system can
limit their abilities relationally. To do this the social model defines
impairment as ``any physical or mental limitation'' (Ibid, 7) and argues
that ``disability signals the social exclusions based on, and social
meanings attributed to, that impairment'' (Ibid, 7). Kafer sums it up
simply: ``impairments aren't disabling, social and architectural
barriers are'' (Ibid, 7). This is in many ways generative for
re-approaching disability's ``problems'', moving them from the
individual to societal, infrastructural and political scales.

With Susan Wendell, Kafer queries the social model by asking, ``how far
one must be able to walk to be considered able-bodied'' (Ibid, 7). While
seemingly simple, this question opens many political and relational
queries. What are the geographies (are there smooth walkways on flat-ish
land?), economics (could they afford the walking aid or pain relief
tablets?) community relations (was their friend free and able to help,
or community carer available?), politics (are their medications legal?)
and infrastructures (were they able to connect to a community, was an
input valid on a form?). These questions move the approach away from a
standardised test on an individual, and into understanding the nuances
of disabling political dynamics on people's abilities, relationally and
societally. Wendell and Kafer consider both impairment and disability to
be socially constructed, as bodies are only seen as disabled in relation
to sedimented ``able'' ones. Returning to Ahmed, we could ask how Crips
line up to the determinate straight able-bodied line or axis? And Crips
would probably respond, ``huh? what axis?'', or, maybe on another day,
feeling more provocative, ``We don't have to!''

Kafer also notes that the social model, when taken to an extreme,
inverts the orientation of the medical model, and in doing so can
invalidate people with disabilities. An example is of those disabled
people who acknowledging their pain or other impairments may have a
desire for a cure to them, but this in the social model could be read as
playing into the medical models' narrative of a cure they are trying to
counter. In this inversion it can also overlook the fact that some
disabilities, such as chronic health, pain and fatigue, will not change
no matter how much we change social relations. Along this thread the
social model also negates to contemplate how a cure narrative towards a
particular (abled) kind of body can affect people who might not identify
as disabled or be ``impaired''. Kafer gives the examples of anti-ageing
creams that can play on the anxieties of getting old as well as growth
hormones for shorter children to make them ``normal'' height. Both can
be understoon as ableist and reliant on a curative model. From this
orientation, we can understand how the social model orients disability
to be delineated along fine lines, and in many ways undermines the
radical knowledges of disabled experiences and the relations they can
put in reach.

Kafer proposes the political/relational model to reorient the social
model's impairment/disability axis. Kafer's model asks what immediate
changes to relations not only validate, but also ease and make room to
build up the capacities of people living with experiences of disability,
neurodiversity, chronic illness and pain ``under any circumstances''
(Ibid, 41), which I go on to discuss further in the next section on Crip
Time \& Crip Futures.

In action, navigating disability through these models can be quite sad,
as we understand how overwhelming these social and institutional
politics and pressures are on our crip bodyminds, and how violent,
forceful and invalidating they are in action. However, by making room to
navigate and analyse these institutional tables and their bodily
horizons from a crip orientation, we are able to set out our own paths
to move around the sinkholes and vacuums they hold in place to erase our
lives, experiences and voices from the discourse about disabled people
and our place in the future. In this research this takes the form of
questioning how I have collectively organise around and with technology,
away from curative narratives of efficiency and control within these
institutional politics, and instead towards collective care, maintenance
and access towards our own crip social and technical infrastructures.

\hypertarget{crip-time-crip-futures}{%
\subsubsection[Crip Time + Crip
Futures]{\texorpdfstring{\protect\hypertarget{anchor}{}{}Crip Time +
Crip Futures}{Crip Time + Crip Futures}}\label{crip-time-crip-futures}}

``Can we tell crip tales, crip time tales, with multiple befores and
afters, proliferating befores and afters, all making more crip presents
possible?'' (Kafer 2021, 418)

One of the focuses of Kafer's \emph{Feminist, Queer, Crip} and her wider
research is her figuring out of crip time. Crip time is manifested by
many different people in crip theory, from Ellen Samuel's \emph{Six Ways
of Looking at Crip Time} (2017), to Margaret Price's \emph{Mad at
School} (2011, p.~62) which offers generative affective definitions, to
Kafer's \emph{After Crip, Crip Afters} (2021) and David P. Terry's
\emph{Explanation not Excuse} (2016) that poetically describe crip
times. \emph{Prioritizing Crip Futures} (Abrams et al., 2024) and Sins
Invalid's \emph{Crip Kinship} of Sins Invalid (Kafai, 2021, p.~75) also
offer crip time in action in institutional and crip collective contexts.
Crip time here makes room for a paradoxical and non-linear temporality
that aims to sit with disabled people's bodies and experiences on their
terms instead of enforcing a normative able-bodied orientation of
time\footnote{``rather than bend disabled bodies and minds to meet the
  clock, crip time bends the clock to meet disabled bodies and minds''
  (Kafer, 2013, p.~27)}. In essence, crip time is the refusal of
determinate, straight, evenly-paced timelines. I choose to build from
Kafer's definitions because of her emphasis on the systemic and
political relations of temporalities that freeze and hold crips and
disabled folks in place, as well as the ways we can hack and form wiggle
room within these dominant hegemonic times. Kafer builds up her
definition through a few moves but starts by cripping Lee Edelman's
queer time against the future (Kafer 2013, 28). Here she takes up the
figure of the `Child' as a symbol of the future that in queer theory is
often also a symbol of the deterministic straight line of inheritance
towards able straight futures\footnote{``The Child through whom legacies
  are passed down is, without doubt, able-bodied/able-minded'' (Kafer
  2013, 29)}. The figure of the Child is also heavily bound to the
histories of eugenics, which conflated disabled, racialised, poor and
indigenous bodies into the categories ``feeble-minded'' or
``defective'', and used to justify forced sterilisation or euthanasia
for the future of the ``race'' (Ibid, 30). The legacies of these eugenic
logics persist in contemporary prenatal selection and abortion of
disabled children to ``solve'' the problem before it starts, and which
the UK has re-enitiated this year (Maia Davies 2025). This emphasis on
the future reinforces the endless deferral of care for disabled folks,
that instead of making room and capacity to care for them now, orients
funding to cure society and futures of their disabled bodies. One
example Kafer takes up is of Noam Ostrander's interviews with young
black men in Chicago with violently acquired spinal cord injuries (Ibid,
33). She highlights this case as it not only shows how the violent
relations through which these men became disabled are so sedimented that
their disabling was treated as inevitable, but also that becoming
disabled was a foreclosure on their life, and one that determined them
to have no future. Here she is sharing how disability is imagined by
ableism to have no future, and that to authoritarian eugenics logics,
reinforced through race, class, sexuality, gender identity and more,
determine disability as a category to forcibly place bodies that they do
not want to see in the future.

In her book Kafer engages with the figure of the ``Child'' to disorient
dialogues around disabled bodies that do not seem to fit in hegemonic
futures. She does this by inquiring into both disabled peoples abilities
to procreate more disabled bodies (children), as well as exploring the
capacities disabled people have to procreate culture, social relations
and intimacies. By doing so, she questions the hegemonic terms through
which disability's futures are represented. She elaborates on this by
studying both mainstream media narratives playing out these political
imaginaries and contemporary social and technical infrastructures that
reinforce these ableist politics systemically.

For example, Kafer reads Marge Piercy's \emph{Woman on the Edge of
Time}, and its legacy within feminist studies as a text that
demonstrates ``inclusivity'' and political democracy within science
(Ibid, 72). Through a crip lens, Kafer's analysis shows it is anything
but inclusive or democratic. This Utopian world has in fact eradicated
all disabled people without any mention or dialogue that included them.
The fictive society is in many ways Utopian because it has eradicated
disabled people through genetic science and technological advancement.
The fictive infrastructure that has enabled this is called the brooder,
which mixes and selects genes to form neutral healthy children of all
races, but in doing so eradicates disabled bodies without question. In
the same move it also blends and displaces race as a way to cure us of
cultural difference and racial inequality. When held in comparison to
the depicted Dystopia, which is filled with disabled and mad bodyminds,
it shares how disabled people's existence here represents one of the
main signifies of this utopian/dystopian split. Beyond the macro of this
split, the only mention of disability in the utopia is of people taking
mental health breaks, where they drop out from society. This in many
ways still enacts the erasure of disability from society, put to one
side, siloed and hidden from sight until normal or cured. Through
Kafer's analysis she shows how neither the utopian nor dystopian futures
make room for or imagine disabled people to live full lives, or affirm
their experiences. In these futures disabled people are either the
abundant problem or are silently cured from existence as part of the
solution. In these narratives, disabled bodies, feelings and experiences
are not included in the making of futures. In utopia, they have been
invalidated and backgrounded at every step towards this ``greater
good''.

In response I have shared how Kafer constructs a crip time that makes
room to hold the affective, embodied and experiential relations to time
that disabled bodies have the capacity to feel. This affective crip
relation to time emerges through disabling relations to institutions,
systems and infrastructures, or just how disabled bodyminds queerly move
through and experience time and space\footnote{``How can i articulate a
  queer crip time that does not oppose queerness to longevity yet
  maintains a critical stance toward hegemonic expectations of
  (re)productivity? Or, to put it differently, how do i respond to the
  fact that the theories we deploy, the speculations we engage, play out
  across different bodies differently?'' (Kafer 2013, 44)}. Kafer
presents many forms of crip time, from the slowing down and indefinite
postponing of care, to the immediacy of when we need to care right now.
Crip time takes the form of cycles of time that complicate the normative
straight cycles of the able-bodied clock, and disables people from
accessing in-flexible institutions, financial stability and
self-determinacy.

Kafer aligns crip time with Jack Halberstam's discussion of queer time,
in particular the ``strange temporalities, imaginative life schedules,
and eccentric economic practices'' of queer life (Ibid, 35). For
Halberstam,the immediacy of the AIDS crisis, specifically for gay men
and their communities took them out of mainstream and normative time, to
a time of ``the here, the present, the now'' (Ibid, 35). In crip
contexts, Kafer describes a ``falling time''. Falling ill or getting a
prognosis changes perceptions of time, shrinks futures and intensifies
the present. When falling into depression, time is slowed to a stop, yet
can fly by. How does time feel when there are uncertainties as to
whether a chemical, food or unpredictable trigger could induce symptoms
that will take you out of time? These fallings into ``strange
temporalities'' leaves space for disabled people to feel these times as
they are and not line-up to a straight and normative time, validating
and enabling crip bodies to tell stories that can change how we think
with time.

For ``imaginative life schedules'' Kafer brings to the forefront the
ways that disabled people have to pre-plan and hack systems of care to
be in reach of their needs. How partners and friends need to split
carers and medicines, as systems have valued one over the other. How
families and friends need to fill the gaps. How bathroom visits have to
be pre-planned and scheduled. Kafer poses these ``imaginative life
schedules'' not just as the extenuated day to day planning that most
disabled people take on, but as the need to project crip bodies into the
future to maintain them. This maintenance is highlighted not just as
(re)production that we see within efficiency models, but as pleasurable
acts of each day, and towards unknown practices of radical access. These
schedules then become a place with a possibility of joy for organising
and maintaining embodied futures for disabled lives that resist the
violence of institutional and hegemonic determination of disabled
futures.

With ``eccentric economic practices'' Kafer talks about the ways that
crip people are economic hackers. When the lines of systemic and
institutional care don't hold you and you start to slip through. When
they limit how much you can earn to maintain your care subsidies and
determine you into a certain level of (dis)comfort. This is when these
eccentric economics are necessitated and practised, to trade goods,
favours and pleasures under the radar so that crips don't compromise our
basic needs and care. These economics for Kafer this offers a blurring
of the public/private in a way that makes room for these economics to be
a test bed for us to imagine what crip politics in action might be
capable of. This is where our everyday practices of collective access
and interdependence become radical methods for rethinking the politics
of how we come together and table, or put to bed our differences.

Kafer's situates crip time between a time of falling in the here and
now, meanwhile maintaining our bodies to enact a future with us in it by
practising politics of our everyday. Kafer goes on to examine how this
framing of crip time lines up against narratives of longevity, focusing
on cripping Halberstam's notion of keeping going ``under any
circumstances''\footnote{``{[}W{]}e create longevity as the most
  desirable future, applaud the pursuit of long life (under any
  circumstances), and pathologize modes of living that show little or no
  concern for longevity.''(Kafer 2013, 41)}. In a queer refusal of
longevity, hegemony is understood to reproduce deterministic futures
indefinitely. A crip position instead sits with each situated body
differently. Kafer asks what it means for that or this body to keep
going ``under any circumstance''\footnote{``i read''under any
  circumstances" and hear ``extraordinary measures,'' ``breathing
  through a machine,'' ``dependent on others.'' i read ``under any
  circumstances'' and hear ``better off dead'' and ``life not worth
  living.''" (Kafer 2013, 41)}, and how can we ease the institutional
and social pressures that curtail crip life at every turn. Here she is
moving to also step out of line and inquire, ``what are the
circumstances we need but also desire?'', and by asking this, also
asking how to orient them to be within reach.

To make time for crip time, where we feel our crip bodies maintained in
a future that holds them is to radically reconfigure our relations to
almost every part of our lives under efficient time. To embody a future
with us in it is to practice access as affirmation and to make room and
time for people to consent, to be present in and the capacity to care
for that time. A future embodied by crips would aim to foster time from
a another radical set of roots that can overlay, intertwine and embrace
a plurality of times. This radical embrace of crip time makes room
within relations, futures and immediate actions that make time for us to
hold this crip stuff with care ``under any circumstances''.

To tie down and bring in-line the proliferating befores and afters of
these indeterminate crip time tales, as I have had to for this thesis, I
have thought about how to find orientation within space and time that
shifts, trickles and lets loose without reason. To sit in disorientation
is to be crip, and to be crip is to be more than aslant from the
straight axis of time. But with a crip disorientation as an axis, one
that changes with bodies in relation, and holds more or less than the
x,y,z of things, I start to trace the paths of my crip time tales to
future horizons abundant with crips. In this research it has meant to
hold slow\footnote{As opposed to hold fast.} to this crip clock, and
find coalitions and collaborations where I build up wiggle room and
flexibility within these hard systems and times for the ``strange'',
``imaginative'' and ``eccentric'' crip practices that make room for
futures for disabled bodies to inhabit.

\hypertarget{crip-intersectionality-coalition}{%
\subsubsection[Crip intersectionality \&
Coalition]{\texorpdfstring{\protect\hypertarget{anchor}{}{}Crip
intersectionality \&
Coalition}{Crip intersectionality \& Coalition}}\label{crip-intersectionality-coalition}}

The Crip relation to intersectionality is one of the most radical
approaches to the term, as it is so abundant and overflowing. This is
due to the variety of bodies, experiences and struggles that exist under
the broad terms of disability. Of course we can turn to Kimberly
Crenshaw's definition of intersectionality(Crenshaw 1991; 2015), and
it's emergence from groups like the Combahee River Collective, where
``interlocking oppressions'' form at the cross roads of identities. This
in many ways comes from and speaks to disability (even if it is not
overtly stated), but I find it more interesting to look at definitions
that come from the overt practices of crip studies. This Crip kinship
(2021) as Shayda Kafai would term it, is where the abundance of crip
communities and interdependence of intersections shine. Kafai's book
centres on the practices of Sins Invalid, a crip performance group
started by Patty Berne and Leroy Moore. Sins Invalid's work centres
expressing crip pleasures, joy and desire without shame and through
interdependence. The group's definition of intersectionality is one that
I hold dear in my research and repeatedly reflect on in my practice.

``Sins Invalid defines intersectionality this way:''Simply put, this
principle says that we are many things, and they all impact us. We are
not only disabled, we are also each coming from a specific experience of
race, class, sexuality, age, religious background, geographic location,
immigration status, and more. Depending on context, we all have areas
where we experience privilege, as well as areas of oppression\ldots{} We
gratefully embrace the nuance that this principle brings to our lived
experiences, and the ways it shapes the perspectives we offer."" (Ibid,
31)

This definition of intersectionality by Sins Invalid has helped me to
think through how I can come together with collaborators to practice
crip freedoms together. To treat these intersections with care and
situate our capacities, privileges and desires into and out of
alignment. Kafer also notes Robert McRuer's notion of ``the non-disabled
claim to be crip'' (2013, p.~169), which I follow: the possibility of
able bodied crips, of those working for the disabled cause, but also
maybe those that recognise the ``the disability to come.'' Committing to
this wider scope of intersectional disability justice is a key step for
any crip studies critique, as accessibly explained in Ella Parry-Davies'
\emph{Coming out Crip} (2022) audio piece. She, much like myself, is
only ``stable'' because we live in the UK with a state health system
that recognises our needs, and can ``treat'' them in its medical scoping
of our conditions. If we were to live elsewhere, have different
nationalities and the relational support those bodies have, we might not
even be here, or able to make these statements. With this we embed
ourselves within the mutual cause against all intersecting injustices
towards disability.

In this dialogue privilege and oppression are situated and negotiated
both with intimacy. With this I have asked how can I as a white crip
researcher, within academic institutions that are often inaccessible to
other intersections of disabled people, use my privilege and oppression
in this experience and context to leverage resources and change
discourses to position disability justice and radical access as the next
action I take at every turn? It also asks me to know that I will never
know the experiences of other people with disability and to have
patience and commitment, to get in touch with and to learn to
communicate our collective sense making. It is to take our sweet crip
ass time to get a feel for one another, and affirm each other along that
path and journey.

I find it important to think through intersectionality with Kafer's
reading of Ashley-X's story. This is because it offers clearly the ways
we have to defy the urge to divide disability through its diversity, but
to instead centre and care for these bodies on the margins of our
discourse. Ashley is born with ``static encephalopathy'', that leads her
doctors to state that ``her development never progressed beyond that of
an infant'', and means that she depends on constant care for her
lifetime. The nickname by carers for the condition and similar
disabilities is a ``pillow angel''. Another element of this relation is
that her body developed at a ``normal'' pace, and at age 6 her parents
started to be concerned about what puberty would do to her. From
consultations with medical ``experts'' they set up a plan to remove
Ashley's uterus and breast buds, as well as put here on a high dose
oestrogen regime. The medical ``experts'' in dialogue with the parents
validated this treatment for many reasons, from ``saving'' Ashley from
pains of menopause and puberty, as well as keeping her small so her
parents could care and protect her directly, as well as keeping her safe
from her own mature body. This meant that the way Ashley and her body
were unconsensually treated was to be shaped for the able bodied world
around them. They just made her easy to handle as well as protected from
relations held in place by ableist and violent sexist politics. I work
with Kafer's reading of Ashley for a few reasons, partially as it
highlights the ways disabled bodies and experiences are curtailed to fit
the hard ableist systems, scales and needs, instead of considering the
disabled bodymind's own needs, orientations and desires. It also shares
the length that these sorts of ableist actions can be taken to, how far
an individual who cannot oppose can be shaped to fit the efficient line,
instead of questioning the political relations of the line itself.

This is highlighted by some mixed views from both medicine and
disability studies, whether a person like Ashley-X is too disabled to be
disabled, and whether people like them need their own category (Kafer
2013, 67). This division provokes yet another boundary beyond the
able/disabled projected axis, forming again new complex parameters that
permit questionably dehumanising and violent treatments, such as the one
Ashley-X recieved. Kafer asks here what if we place people like Ashley-X
not outside of disability discourse, but in the centre of it? When I
start to imagine relations and tables to care for these experiences and
not determine and invalidate their communication ability, I open myself
up to feel and be felt through many yet unknowable ways. Kafer's
enactment of crip theory here also centres the joy of this
communication. In Ashley-X's case, a double mastectomy and hormone
blockers to stop her breasts growing means that Ashley was also disabled
from having any pleasure from these removed organs. Their pleasure was
never imagined, nor valued in their future, only pain. That just because
these experts were did not have the capacity to feel Ashley's
expressions of joy, imagine their bodily possibilities of pleasure and
agency, or even contemplate their political desires meant that it was
these experts who had invalidated Ashley without question or dialogue.
Instead of joy or pleasure, the dialogues around Ashley's body focus on
pain or discomfort and keeping her secure from strangers, who may also
feel this discomfort around her. Reflecting on Ashley-X through crip
intersectionality, it can also be understood that this probably only
happened to her as she came from a middle class American family that
could afford to treat her through these medical industrial institutions.
When I highlight her context, I think to how I can build up capacities
to be in dialogue with each other. This is also amplified by disability
as such a diverse field that includes so many bodyminds in so many
different situated social/political relations. Again a crip move is to
centre joy and pleasure as an actionable site for us to move from, but
this is not to say we silence or ignore the very real pain that exists
everyday within our community internationally and intersectionally. It
instead aims to move beyond deterministic narratives of disability as
painful and awkward and asking what brings us together through joy,
comfort and pleasure, even if we are dealing with pain, depression and
disabling social relations. Through feeling out these restraints we can
start to move towards interdependently asking/demanding how to put our
desires in reach of one another?

In Kafer's final chapter, `Accessible Futures, Future Coalitions', she
gives an example of Silvia Yee's work with the Disability rights
education and Defense Fund (DreDF) (2013, p.~160). Kafer highlights
their work with communities facing the overburden of toxic industries
and emissions to show how policy and access put in place to help
disabled people in many ways benefits everyone. In the case of DreDF the
rights of children with respiratory issues can stop schools and
institutions from using potentially harmful chemical pesticides, which
can benefit not only the quality of life of everyone in that district
but also reduces the ecological impacts. This presents Crip
intersectionality as a powerful place where in coalition we can leverage
our communities access privileges to not only improve their individual
lives but to put a future we all want into reach.

This is where the interdependence of crips also sits as an essential
practice. When disabled people struggle to perform as required by a
system or to be valued the same as able-bodied people, communities have
to help each other manage the gaps. Like I shared in Crip Time \& Crip
Futures with Kafer's take on Halberstam's ``strange temporalities,
imaginative life schedules, and eccentric economic practices'', all of
these crip times are threaded through community schedules, care and
interdependence. These acts can range from just consolidating to each
other that it is fine that we are sad and mad, that there are gaps in
what we are able to do, and being there when we almost fall through
them.

In the third episode of the Crip Club podcast\footnote{\url{https://thecripclub.co.uk/podcast/}}
, which discusses disabled representation in films and media, Clare
Baines interviews Jessi Gutch around cancer portrayals (Baines, o.~J.).
In the podcast they not only talk around what it means for Gutch to not
fit into the curative narrative of cancer, but she also shares her
experiences of living with a disability whilst working in the film
industry. The section I felt was most informative for understandings of
intersectionality and interdependence of disabled folk, is when she is
talking about working in a majority disabled set. This crip background
informed a very different set of expectations and capacities for one
another. Gutch says that this formed a space where people could express
their needs easily and have them affirmed, where everyone on set knew
that this film was important, but not the most important thing. The most
important thing is that everyone in this collaboration was feeling as
good as they could in their bodyminds and able to do what they wanted to
within the collaboration. It was where if they needed more time for this
crip stuff they could have it without question.

In my practice as I have started to collaborate with more disabled and
crip people I have also felt this same energy, where our wellbeing,
capacities and feelings are prioritised over an imagined deterministic
outcome or performance of work. I talk more about this in
\href{../02_Crip-Tic\%20of\%20vignettes/02.01.04_The\%20crip\%20table.md}{02.01.04\_The
crip table} section, but to touch on one of these collaborations that is
of note here was in the creation of the crip infrastructure en-crip-ing
time\footnote{\url{https://time.cozy-cloud.net/}}, a work that imagines
and instigates crip times within computational time. I go on to discuss
this work in context and in more detail in a later inquiry discussed in
\href{../../06_A\%20Cozier\%20Configure-Ability/06.03_En-crip-ing_cozy_time.md}{06.03\_En-crip-ing\_cozy\_time}.
Here though one major part of the work was to focus on this type of crip
relating between my collaborator Mariana Marangoni and I in the making
of the infrastructure. This meant that there were no agile time-lines,
roles or outcomes here, just crip times, flexibility and compassion. We
both had many demands from life, and very crip times in relation to our
bodyminds, but we both understood that for this intersection and crip
infrastructure to emerge we both needed these times to prioritise our
needs. When one of us had to freeze time to recover or get through
something it was okay, we just said, and adjusted the timeline or
responsibilities. When one got stressed and anxious about the imposed
time-lines, the other didn't combat this, but comforted it. The
expectations would be flexible to fit our current capacities, and
frictions became places to orient from and to learn each other's needs
and approaches. These social politics, ways of relating and maintaining
crip infrastructure became important to show in the work. To make it
present in the background of the work we added comment sections in the
html of the poem, showing our conversations of making time and space for
this crip stuff.

Here I return again and again to imagine where we can find joy in
rubbing up against each other's supports and restraints in relation, and
how this practice is a language we all have to practise and build
capacity for independently. It is a language of sensing what each
other's capacities, privileges, knowledges, desires and feelings are,
and learning to understand them as they unfold. In a crip
intersectionality I turn to this movement of coalition and
interdependence in action, of bringing together divergent bodies and
movements to manifest the change we as a group need now. Crip coalitions
in themselves are temporary comings together of differing politics,
structures, goals and needs to configure the social and technical
infrastructure for change and affirmation we need now. In Ren Britton's
Coalition Bouquet\footnote{https://lorenbritton.com/projects/coalition-bouquet},
they trace many of these histories and the improvised technologies,
politics and communities that came together to blossom crip change
together. Here these often complex, divergent and temporary coalitions
can move beyond the predetermining confines of pain, which they may be
in, to where they can reorient the issues of intersectional access to
make room to hold the possibility of joyous direct action and change.

In this research, this has been taken up as working between different
departments, institutions, collective, projects, coalitions, mailing
lists, servers and etherpads. Most importantly though, for me as a
person with fatigue, chronic pain and other complicating factors, this
has meant pushing for accessible ways of managing and maintaining
collective practices, and trying to introduce a disorienting crip axis
to (dis)align with. This inquiry has been about intersectionally and
interdependently reorienting practices, figures, timelines,
expectations, infrastructures and priorities. In particular, inquiring
into how I can question my own role in these dynamics to support crip
politics, both in our organisation and by making them accessible to
collaborators. I am also taking this up in response to broader questions
of interdependence within the networks, infrastructures, and politics
these collaborations manifest and are held in place by.

\hypertarget{cripping-technoscience}{%
\subsubsection[Cripping
Technoscience]{\texorpdfstring{\protect\hypertarget{anchor}{}{}Cripping
Technoscience}{Cripping Technoscience}}\label{cripping-technoscience}}

For ``cripping'' technology, Cielo Saucedo and Nat Decker's
\emph{Cripping\_Computer\_Graphics} (2023) which I touch upon later when
\href{../../06_A\%20Cozier\%20Configure-Ability/06.02_Manifesting_intentions.md}{06.02\_Manifesting\_intentions}
for the Cozy-Cloud server. I highlight this essay here though as they
are working with Dr.~Carrie Sandahl terming of ``cripping'' (2003, 37),
as a way to feel out in action the ableist assumptions and inaccessible
norms of computer graphics and the relations/politics of technoscience
more broadly by bringing embodied knowledges and practices of crips in
touch with them.

Bringing it back to Kafer's work In \emph{Feminist, Queer, Crip}, she is
cripping technoscience by inquiring into the figure of the cyborg and
orienting it around disability on two points of contact/impact.
Initially, she questions how media represents disabled dependence on
technology as real-life cyborgs, leading to the ``rise of the cyborg''
as the unquestionable and naturalised cure for disabled folks. From here
she then moves towards a crip critique of Donna Haraway's \emph{Cyborg
Manifesto}, away from the naturalised cyborg\textasciitilde crip
relations and towards practising relations that challenge the
militarised imaginaries of cyborgs and disabled bodies through a
disorienting and disobedient crip approach. This refusal, not of these
technologies, but of how their imaginaries are oriented, politicised and
enacted, is key to my disobedient crip research into technology.

Kafer's media analysis of cyborgs shows that they are again sedimented
without question to be a cure for impairment, to make `normal' disabled
bodies. These narratives, as raised by Kafer's analyses of Chris Hables
Gray's \emph{Cyborg Citizen}, lead to Gray, an able-bodied science
studies scholar, taking up the late \emph{Superman} actor Christopher
Reeve, without his knowledge or consent, as a heroic cyborg leading the
way for the disabled community. Gray suggests that the ``crippling of
superman'' and Reeve becoming dependent on multiple machines and devices
is one of the best examples of a ``true'' cyborg. Gray states that Reeve
is a hero as he leads the way for other disabled bodies to be cured
through assistive technologies. Gray's text in a number of ways
undermines the disabled experience, as it doesn't make room for Reeve or
people with disabilities, and more specifically quadriplegic people, to
speak to their use of and relation to access and assisted living
technologies. Kafer also points out that Gray overlooks and cannot
imagine that there is a wider disabled community than just the celebrity
in focus, Reeve, or that these communities have a long history and
culture of living with assistive technologies that might be informative.
Kafer also shares how his use of language often refers to people with
disabilities as ``invalids'' or ``patients'', conflating their whole
identity and personhood with an injury or the medical identification of
their condition. Sitting here, Kafer figures Gray as a prime example of
how the disabled bodies are oriented when deployed in cyborg theory. In
one of many other examples, Annie Potts' ``taxonomy of cyborgs'' places
Reeves in a list with fictional cyborgs, orienting him as less human and
more cyborg, a construction rather than a complex, multifaceted person.
In these cyborg curing narratives, Kafer shows how this signalling of
disabled people as the truest cyborgs locks in and has sedimented the
path so much that a cyborg itself becomes a signal for disability,
abnormality and difference. The cyborg\textasciitilde crip, the
crip\textasciitilde cyborg.

By contrast, Kafer reads Haraway's figure of a ``cyborg'' in the
\emph{Cyborg Manifesto} (2006) as problematic in some aspects, but an
empowering plurality to think through when approaching with feminist
STS. The problematic notions of the Harraway's cyborg are somewhat well
known; for example, its homogenising of experiences ny taking an
imperializing approach that dictates one figure of an identity for whole
groups of people (Kafer 2013, 114). Kafer, however, offers a closer crip
critique of Haraway's cyborg, tracing how she has also fallen into the
sedimented route that portrays disabled bodies, or as she determines
them ``paraplegics and other severely handicapped people'', to be
exemplary or true cyborgs. In this move, Haraway makes a number of
questionable choices. She addresses these imagined people with
disabilities in terms of ``handicap'' and describes them as
``paraplegic'', which reinforces the individualism and diagnosis of the
medical model. Kafer also points out this critique of naming languages
was fairly common discourse within that era of disability studies, of
which Haraway was likely aware. Haraway also homogenizes experiences of
disabled people like she does for other identities, but in doing so
displaces the affective relations to technologies these different
disabled positions and experiences make room for. She essentially does
not include any affective perspectives on what it feels like to be a
person who is disabled and (inter)dependent with/on technologies and
infrastructures.

Even though Haraway's ``cyborg'' is in many ways flawed, especially in
relation to intersectional disability, Kafer still finds it to be
generative from a crip position. In particular, the cyborg can also
figure the defiant feminist care in and against flawed and often violent
technoscience. The cyborg finds joy in acts of practising radical care
otherwise. Kafer disorient the cyborg as a fluid figure that can make
room for us to figure out other paths between people with disabilities
and the technologies that support them. She refigures the cyborg as a
place of embodied crip futures, on their/our terms, validating and
tracing a plurality of paths we take with technology together. It is
this focus on how the cyborg makes room for crip studies to cross paths
with other disciplines and experiences that is what makes this a
generative space for her. Kafer gives examples of different cyborg
intersections within disability studies, such as Dean Spade's research
into the medicalisation of transgender identities. Spade navigates trans
experiences of gender identity disorder (GID), which is used to validate
care and medication, but also acts to reinforce the binary of gender as
well as prolong and delay care (Kafer 2013, 124). GID also leads to
tensions, with potentially ableist undertones and a medicalisation of
queer identities, where needing to self-identify and be validated as
having a ``condition'' or disorder to be treated as trans could lead to
gender identity being read as an individual handicap instead of
political and relational issue. Spade also reads this medicalisation of
GID as reproducing the ``curing'' narrative, where specific treatments
can resolve the individual ``problem'' of gender identity, and can be
industrialised. Kafer works with Spade and others here to trouble these
notions by stating ``that neither medical technologies nor diagnoses can
be characterized as purely oppressive or politically neutral'' (Ibid,
125). This disorientation feels the individual within the relational,
situated, and makes room for the cyborg and their bodily horizons to be
read as ``maps of power and identity''. In doing this, the cyborg also
makes space for ``affinity-through-difference''\footnote{Who Kafer with
  Chela Sandoval, lament over the often misplaced genealogy of
  ``affinity-through-difference'' to Haraway, instead of to the
  contemporary indigenous writings (Kafer 2013, 117)}, where disabled
people as a very diverse community can come together and lay with the
disorientation of these many contradictory bodily maps and horizons.
With this Kafer figures crips to be cyborgs not because of their bodily
relations and dependence on assistive technology and technical
infrastructure, that have been naturalised to signal disability, but
because of the political practices we disobediently undertake. These
acts of refusal are the tracings of our own queer and crip lines, the
feeling and coalescing of other crip techno bodies, tables and
relations. Some examples of this kind of rubbing up against and
disobeying technology, whilst living through it, is shared in Laura
Forlano's many works critiquing the cyborg as a concept. In both her
\emph{Hacking the Feminist Disabled Body} (2016) and \emph{Living
Intimately with Machines} (2023) she foregrounds the disabled experience
of technology and relation to it as a place of generative friction. If
oriented this way it can make room for other perspectives, methods and
ethics to be present and felt at these tables. Kafer concludes her
analysis by vulnerably opening up the idea that the cyborg within this
plurality is also not the only figure for disability, as some disabled
experiences may relate more and some less to these technological
entanglements. However, she does encourage attending to these sites
where disability is pressent within discourse but not felt or made room
for. Much like Kafer's intervention into Haraway's cyborg, I also take
this approach later in
\href{../../04_Configure-able_Methods/sections/04.02.02_Cripping\%20Configuration.md}{04.02.02\_Cripping
Configuration} with Lucy Suchmans \emph{Configuration} (2012) to
critique the ways disabled bodies can disorient the horizons and
practices I take toward collective organisation and network
infrastructures.

\hypertarget{crip-technoscience-manifesto}{%
\paragraph[Crip (Technoscience)
Manifesto]{\texorpdfstring{\protect\hypertarget{anchor}{}{}Crip
(Technoscience)
Manifesto}{Crip (Technoscience) Manifesto}}\label{crip-technoscience-manifesto}}

Kafer follows up on \emph{Feminist, Queer and Crip} (2013) with a talk
called \emph{Health Rebels: A Crip Manifesto for Social Justice} (2017)
as well as \emph{Manifesting Manifestos} (2023). In the Health Rebels
talk she covers many of the bases of her work that I have covered
elsewhere, but through other informative anecdotes, experiences and
relations through which to imagine this work. The focus in this talk is
coalition, of cross-issue collaboration and working within difference
together. In the talk Kafer appears to refigure her crip cyborg into a
number of other figures, such as ``health rebels'', ``the checklist'',
and ``the manifesto''. Health rebels play out somewhat similarly to the
cyborg, as the defiant crips in the face of the medical industrial
complex and institutional ableism. In coalition, living crip as method,
multiple and indeterminate, forming frictions at our many horizons and
intersection with these generalised and prescribed norms. Disobediently
disoriented, aligning to the crip axes. The checklist represents the
rubric, matrix or limited drop-down selection that indeterminate
disabled bodies are (in)validated through. I discuss my experiences of
this further in
\href{../../02_Crip-Tic_of_Vignettes/sections/02.02_The_computing_table.md}{02.02\_The\_computing\_table}
section, where efficient logics determine static lines that are used to
invalidate needs, experiences, practises and knowledges. Building on her
earlier troubling of the crip cyborg as the only figure to live by,
Kafer loosens and refigures the manifesto to be a place of wiggle room,
of playing out our dreams, testing out futures, of finding gaps and
frictions in these manifestings as generative places to write other crip
manifestos and imagine other futures from action. To do this she turns
to queer theory, as we could easily to Ahmed, to reflect on how a
manifesto (or an orientation) moves along a horizon, foreclosing and
limiting some relations whilst putting some others in reach. For Kafer,
quoting Audre Lorde, the manifesto is ``believing, working for what has
not yet been while living fully in the present now.''\footnote{Full
  quote: ``One of the hardest things to accept is learning to live
  within uncertainty and neither deny it nor hide behind it. Most of
  all, to listen to the messages of uncertainty without allowing them to
  immobilize me, nor keep me from the certainties of those truths in
  which I believe. I turn away from any need to justify the future- to
  live in what has not yet been. Believing, working for what has not yet
  been while living fully in the present now.'' ― Audre Lorde} Here it
acts as a place to live with uncertainty and practise actioning and
inquiring into crip methods for indeterminate futures.

Kafer later follows this talk up with her essay \emph{Manifesting
Manifestos} (2023) where she takes an even closer reading of this
writing format. In this essay she traces their problematic past within
patriarchal modernist thought that unifies bodies, but also the
sedimented critique of feminist manifestos that call the mad and crazed.
In this Kafer disobediently steps into this critque to question how the
manifesto makes room for this crips to be mad with and through their
madness. With Breanne Fahs, Kafer also proposes this crazed manifesto as
a format that ``invites disorientation'' (2020, 5) as it makes room for
a distortion of time and a projection of bodies into other futures not
yet in reach, but desired now. She also offers with Kafer (2016) how
manifestos are a place to iterate from, adapt, change and find holes in
our plans and manifestings. Here again not invalidating, hidding or
trying to cure these flaws but, as she reflects on with her students
manifesto, it is about making change through embracing and being
oriented through these many situated flaws on the plurality of bodily
horizons I come into contact with. In this research I ask alongside
Kafer how, through a plurality of manifestos, can crip studies and
disability justice can take on struggles we may have never imagined
before? And how can we try to make room for people and relations that
have not yet been imagined or thought of at these sedimented
institutional tables yet?

In this research, I iterate on from (but still with) Kafer's plurality
of crazed manifestos to work more closely with Aimi Hamraie and Kelly
Fritsch's \emph{Crip Technoscience Manifesto} (2019). This is to orient
my research more closely towards critiques of technoscience, but also to
highlight crip manifesting and manifestos in practice. Their manifesto
reiterates Kafer's work in a similar way to what I have discussed here,
but situates it through more specifically STS relevant examples and
positions. Key examples that are relevant to this work include ADAPT
activist forming and shaping physical infrastructures by pouring ramps
and smashing curbs, as well as Alice Wong's \emph{Disability Visibility
Project} forming social and media representation by and for disabled
people. Both examples, and many other crip sociotechnical hackings they
touch on, manifest forms of direct action, of making change and caring
for community infrastructures where institutions invalidate the
political/relational forms through which they perpetuate ableism.
Hamraie and Fritsch orient the \emph{Crip Technoscience Manifesto}
itself through commitments, manifesting a crip technoscience
specifically through direct actions, embodied protocols and political
demands. They do this to not only make this manifesto clear and
accessible, but also to project into the future the embodied actions and
relations that could make room for crip discourses and politics in
practices to flourish in relation to technoscience.

I discuss these commitments below to highlight key orientations this
work is building from.

``Crip technoscience centres the work of disabled people as knowers and
makers.''

This positions the everyday practices and experiences of disabled people
as that which orients them to best know their bodyminds and relations to
technologies and society. They work with a few examples of crip
designers to show how this plays out in action. Alice Loomer, for
instance, shows us how wheelchair maintenance and hacking by the person
using it, even though not a frictionless experience, forms practices and
knowledges of technology that are situated in relation to that body and
its horizon. They quote Loomer saying, ``I have equipment that fits
me'', offering how disabled knowing/making through embodied expertise of
technologies can make room for these bodies within technoscience. This
is not to glorify the Do-It-Yourself practices of disabled people, as
some may not be able (or want) to do ``it'', but it is to uplift the
``world-dismantling and world-building labors'' that ``misfitting'' in
the world as a crip means and to orient from that disorientation.

``Crip technoscience is committed to access as friction.''

Friction for Hamraie and Fritsch represents the refusal to fit in of
crip technoscience, of wanting more than the slots you are offered in
the determined checklist, and taking actions and making frictions to
make that room. They lay with the ``non-compliant user'', from Ray
Lifchez and Barbara Winslow's book \emph{Design for Independent Living}
(1979), who use technology as a friction against an inaccessible
environment to highlight the change needed\footnote{``Illustrating this
  with an image of a powerchair user wheeling against traffic on a
  street without curb cuts (1979, p.~153). This technology-enabled
  movement against the flow of traffic marks anti-assimilationist crip
  mobility: not an attempt to integrate (as in the liberal approach to
  disability rights), but rather to use technology as a friction against
  an inaccessible environment.''(Hamraia and Fritsch 2019, 11)}. They
also discuss Collin Kennedy, an activist and cancer patient, who refused
capitalist time around care by spraying expanding foam into hospital car
park meters. In doing so, Kennedy disabled the capacity of these
infrastructures to monitor and charge patients visiting the hospital,
and so forming new possibilities and imaginaries of crip time through
direct action forming friction on the sedimented norms. When thinking of
frictions, I also want to return to the section on
\href{01.02.03_Crip_intersectionality.md}{01.02.03\_Crip\_intersectionality}
with Ashley-X, and think again about the possibility of curtailed joys
and pleasures that can come from the frictions of our disabled lives.
With crip technoscience this also aims to ask how direct actions,
material transformations, practice of maintenance, the rubbing up
against restrictions, and the frictions felt from these actions can make
room for joy, pleasure and abundance. How could Kennedy wiggling the
nose of the expanding foam can into the ticket machine to make room for
crip time in a car park, bring joy to not only him, but also to others
who were visiting that day. Those who now did not need to rush, or keep
in-line and on time? These friction-enabling experiences of crips slips
beyond institutional and capitalist time and offers glimpses of a future
we can wiggle into with and for one another. With ``access as friction''
we can embrace the struggles of making change where we need it, of
getting in the way, and disorienting conversations and in doing so
reconfiguring the infrastructures and institutions around us, possibly
even with unknowable joy and pleasure.

``Crip technoscience is committed to interdependence as political
technology.''

In the scoping of both feminist technoscience and the
political/relational model of disability, how we work together
around/for/with technology is itself a type of technology and method
that we must orient around and practice. In crip technoscience and wider
crip studies, there is a dedication to practising this through
interdependence, focusing on how our capacities build up to together,
and how we can make room for each other. What politics come from these
places, what emerges from the ground up and can radically change the
ways we imagine, embody and practice potential future politics now? One
example Hamraie and Fritsch quote which resonates with this work is that
of Mia Mingus's ``access intimacy'' (2017), which is hard to describe
but easily felt. Access intimacy attends to how much someone `gets'
another's access needs, how they make room for them to express these
feelings, even if it's stating you don't know what they are right now.
Access intimacy as a concept makes room for me to imagine intimate ways
of relating to others and how we can care for one another's access
needs. It is to be open to the unknown and practice getting to know
other's access needs.

``Crip technoscience is committed to disability justice.''

Manifesting crip technoscience that engages in disability rights goes
beyond the assimilationist ideologies of mainstream accessibility to
highlight how interconnected struggles, and finding justice in them, can
make room for even more radical practices for crips within
sociotechnical systems. When working for disability justice, crips
refuse painful assistive technologies which only normalise their bodies
to make others more comfortable. Disability justice also aligns this
work with many intersections of disability studies to form a crip
technoscience that is informed by and practices with working class,
black, brown and indigenous disabled knowledges. It is one where we
vouch for crip wholeness and life affirmation as places we can radically
challenge the economics and politics of ableist powers that see us as
broken or invalid. For this holding the impossible they turn to Eli
Clare (2017), whose work suggests we can be broken and yet whole,
imperfect, but valued, and holding a generative tension between the
(not) opposites of these things.

Hamraie and Fritsch also engage with the work of Sins Invalid and other
crip artists and hackers to think through what Alice Sheppard (2019)
calls ``cultural-aesthetic technoscience''. This is where they bring
together new assemblages and aesthetics for communicating these complex
relations of wholeness. Sins Invalid, for example, combine performance,
technological aids and films doused in unashamed crip joy and pride, not
hiding, but embracing and putting on show the potentials for love, grief
and playfulness in a very serious but joyous space. As these thinkers
and artists share, joy and pleasure are not the only way to achieve or
approach disability justice as they can undermine the complex relations
of disability globally, which work through oppression and debilitation
as politics. Instead, engaging in and making room for this contradictory
wholeness of disability justice is what shapes a crip technoscience that
coalesces and dances with these sedimented tables through resistance. It
is where I can build from the non-innocent, from the disabling
continuation/aftermath of war, Big Tech extraction, petrochemical and
medical industrial complex, to work with these embodied and plural crip
experiences, knowledges and practices to form other ways of caring for
each other, our technologies and our communities.

Crip manifestos are multiple in the ways that we can understand them
coming together through action. They can move from stating principles
for their organisation or community, such as Sins Invalid's \emph{10
Principles of Disability Justice}\footnote{https://sinsinvalid.org/10-principles-of-disability-justice/},
Johana Hedva's \emph{Access Rider} (2019) and Healing Justice London's
\emph{community agreements}\footnote{https://healingjusticeldn.org/methodology/community-agreements/}.
They can also take up imagining infrastructural needs and transforming
norms to a crip axis, which we can see with MELT's \emph{Access
Server}\footnote{http://meltionary.com/antiableisttech/accessserver/} as
well as Carmen Papalia's \emph{Pain Pals}\footnote{https://leonardo.info/criptech/eaat/pain-pals}
and many mentioned earlier in this section. They can also take the form
of literary fabulations and manifestos, such as Kevin Gotkin's
\emph{Artists in Presidents} speech encouraging us to get
cozy\footnote{https://www.artistsinpresidents.com/kevin-gotkin}, Ren
Britton and Helen Pritchard's call \emph{For Careful Slugs: Caring for
Unknowing in CS (Computer Science)} (2022) as well as Sophia Maier, V.
Jo Hsu, Christina V Cedillo and M. Remi Yergeau's call for us to
\emph{Get the Frac In!} (2020). These crip forms of manifesting, of
making principles and agreements, of forming and imagining
infrastructures and room we need now, and of telling stories of futures
and relations with us in and yet to unfold, are what this research
orients from. It inquires through how In my collaborations we can
collectively imagine and action these other sets of politics, protocols
and ways of relating from the sites of impact and friction. It is to
make room for us to feel out from our embodied disorientations and
towards community led technology. In this movement I have taken up
disobedient action research as my methodology, as the next chapter
highlights. In this work, my research is performed across scales, from
internally inner arcs of embodied change and practice to movement that
aim to go beyond by becoming intimate with these institutional limits
and norms.

\hypertarget{conclusion}{%
\subsection[Conclusion]{\texorpdfstring{\protect\hypertarget{anchor}{}{}Conclusion}{Conclusion}}\label{conclusion}}

This chapter has shared how my research is orientated through a crip and
queer approach that aims to trouble the sedimented ableist norms of
computing practices and technoscience through a disobedient queer axis
of disorienting collectice Crip manifestings. With this approach, the
research makes room for my collaborators and I to be in touch with these
network infrastructures. Doing this we focused on where we felt slips,
slants, gaps and frictions to orient our approaches to the normative
logics within computing. This was to question how these points of impact
give us an ambivalent relation for us to imagine and practise crip
futures from now. In coalescing Ahmed, Kafer, Hamraie and Fritsch, as
well as many others, together in a crooked crip line for this chapter, I
have formed a framework that aims to overflow from these sites of
friction. This is to not avoid or invalidate these disconnects and
misfittings, but to think through how they can inform other orientations
from a position of radical Crip intersectionality, interdependence and
disability justice. Working from these crip slants, naps and f(r)ictions
are places that can build up our collective capacities for us to move
off from the sedimented straight line of inheritance, break open the
normative pipeline, and its determined relations, and make room for this
research to question how my collaborators and I want to approach
relational ``problems'' we are seated at together ourselves.

To disorient these tables and relational norms I later turn to the
\href{../02_Crip-Tic\%20of\%20vignettes/02.01.04_The\%20crip\%20table.md}{02.01.04\_The
crip table} as an axis of radical access, interdependence and coalition
where I informed my practise and research. Moving towards this crip
table from the other sedimented norms of
\href{../02_Crip-Tic\%20of\%20vignettes/02.01.02_The\%20computing\%20table.md}{02.01.02\_The
computing table},
\href{../02_Crip-Tic\%20of\%20vignettes/02.01.03_The\%20operating\%20table.md}{02.01.03\_The
operating table} and
\href{../02_Crip-Tic\%20of\%20vignettes/02.01.01_The\%20research\%20table.md}{02.01.01\_The
research table}, I orient to move away from these ableist and violent
relations, and towards infrastructural politics of crip care and
affirmation. This crip axis makes room for this research to take on
coalition, collective practice and of movement making as a core
practice, thinking through how social relations of infrastructures are
political technologies that we can coalesce (in) together.

Alongside
\href{../02_Crip-Tic\%20of\%20vignettes/02_Crip-Tic\%20of\%20vignettes.md}{02\_Crip-Tic
of vignettes}, this chapter aims to provide fertile methodological
ground for crip manifestings to be approached through disobedient
actions and within computing. This methodology and its iterating
plurality of manifestings will unfold more throughout the thesis, but
moves to put more digital network practice within reach of crip theory
discourse. This is to think about how we can foster and centre crip
practices and politics of radical care within our local community and
collective configuration practices, but also to think about how these
politics can overflow and make room and flexibility in these hard
systems for other lines to be emegred elsewhere by others also facing
their own frictions with efficient, white supremacist and curative
politics. This approach aims to form collective politics that do not
determine or prescribe methods and approaches, but makes room to
contemplate how I as a collective technical practitioner can become
intimate with collaborators access needs. Here making wiggle room for us
all to be experts, practitioners and knowers/makers of our own (life
affirming) infrastructures, and in doing so gaining agency and
capacities at the tables we have always been present at, but were never
allowed to get to grips with.
