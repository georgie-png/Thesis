\hypertarget{introduction}{%
\section{Introduction}\label{introduction}}

This thesis brings to the table my research into and practices of
Configure-Able methods. These Configure-Able methods have emerged
through my researches transdisciplinary approach, which brings together
Queer Feminism, crip theory and Science and Technology Studies (STS).
With this research approach I inquire around how to form wiggle room
within the inherited norms of network and organisational infrastructures
when configuring them out locally with communities. In this context I
collaborated with a Trans*Feminist digital arts collective
In-grid\footnote{\url{https://www.in-grid.io/}}, and the initiation of a
crip centred server space called the Cozy-Cloud\footnote{\url{https://cozy-cloud.net/}}.
With this question of ``is it Configure-Able?'' I inquired with them
into how our communities, digital art collectives and crip servers can
manifest their social and technical infrastructures to fit around our
capacities, bodies and needs. In these inquiries we found friction with
the determined roles, relations and matters we rubbed up against or were
forced into. Here we orientedto trying to put our communities and their
organisation/network configuring practices in touch with each other by
distributing expertise through critical access, so that they can
mutually affect the capacities, limits and horizons of each other. With
my inquiries this has both oriented the configuration of network
infrastructures and organisation practices to be more accessible to
collective practices and dialogues, but also for these infrastructuring
and organising practices themselves to be figured out by community
impacted by them. This is where my work with critical access and crip
theory disorients this research's approach and demands not for extra
time or room around the inherited norm, but for community,
interdependence and impact centred practices to become an indeterminate
norm that we centre on when approaching configuring our collective
infrastructures.

Configure-Able methods in this research makes wiggle room with
communities by disorienting the figures of their social and technical
configurations. Figures are the roles\footnote{User, Designer, Server,
  Client, Expert}, stories\footnote{Of zero trust internet networks,
  security as privacy and centralised control as achievable.}, and
metaphors\footnote{Secure Shell (SSH) handshakes, firewalls and chronic
  jobs.} that orient and lay out the plans that go on to be configured
into infrastructures. So when, with Configure-able methods in my
collaborations, when we start to disorient their figures from the ground
up, such as users as carer, servers as collaborators and security as
safety through intimacy the plans and their configurations drastically
change. This is especially emphasised in my turn to radical disability
politics of crip theory and its disobedience that goes beyond the
talking about frictions we have with the figures and politics, to
reorient to making friction through direct action and configuring out
our own local improvisation together where possible. This direct action
in my collaborative inquiries have even been simple things like
admitting our collective orientation and ethics is out of reach for now.
In this acknowledgement though, is a figuring out together what we do
want and how to wiggle towards the room this needs together. These
actions have also been more material movements such as collectively
manifesting In-grid's first server infrastructure, as well as forming a
ccrip centred server called the Cozy-Cloud. Through these inquiries and
reflections I aim to offer up forms and capacities that these
Configure-Able methodology not only enact, but demand!

Later on in this chapter I go through a closer reading of the thesis and
give an overview of each chapter to orient them to be more accessible.
Next though I go through figuring the background of this research, which
is a recurring move in this thesis. Bringing focus to the background
gives room, contour and shape for figures and bodies to emerge from. In
this initial background I give a short introduction to why my research
and practice disobediently oriented towards the emergence of
Configure-able methods through collective and community centred
organisation and infrastructuring.

\hypertarget{background}{%
\subsection{Background}\label{background}}

In a way this background to the research is so far away and so small on
its metaphorical horizon that it only takes up this small section of the
introduction. I still find it important to mention these now distant
inquiries as it offers many of the reasons for why these Configure-able
methods, my collaborative inquiries and this thesis have come into
being. Coming into this PhD I was very much following the norms of
computer science, and even though I had been tought to code creatively I
had still been taught how to optimise code, automate decisions and of
course train AI models to make them for me. With this sedimented norm in
place, even though I was fairly set on collective knowledge creation, I
of course started by making many different bits of code, databases, and
critiques of how AI might do things with us or for us.

As I proceeded in this inherited practise and normalised path I started
to read up and relate more to crip theory and its critiques of
technology. One of these critiques that surfaced was of AI and
computations histories and practices that originate from those of
eugenics. Eugenics as a phony science believes that we can shape and
cure the human race of unwanted and feeble minded individuals through
genetic selection, and has a long history of violence toward disabled
people and many other marginalised groups. Wendy Chun in here book
\emph{Discriminating Data} (2021) covers how these algorithms work on
both mathematics and social levels through not only these eugenic
logics, but also segregationist algorithms that split people into groups
via different physical and social ``features''. Timnit Gebru and Émile
Torres in their essay ``The TESCREAL bundle'' (2024), and as part of
D.A.I.R.\footnote{https://www.dair-institute.org/}, also bring together
current histories of how eugenic genetic curing have also been
accelerated into silicon valley transhumanist imaginaries that cure
humanity of our feeble and disposable bodies as a way to reach a ``net
good''. To find this cure they are of course gambling everything on the
yet out of reach Artificial Generalised Intelligence (AGI) and its
unknown and often undeclared capabilities. Johannes Bruder in his
chapter \emph{AI as medium and message} (2023, 171) where he discuses
how the imaginaries of AI both fetishise and want to cure Autistic and
neurodiverse individuals in particular.

Later on in the flipping table (chap) I go on to talk more closely about
crip takes on technology, and Alison Kafer's (Kafer 2013) crip critique
of transhumanism, and this crip refusal of the imaginary and need to
cure people through these ever unferling and never in reach speculative
future technologies. In response Kafer offers a refusal of a future cure
to instead to orient to what is in reach. In changing this focus, Kafer
is asking how can communities form social and technologies politics and
relations that can manifest the care and affirmative infrastructures
now?

When inquiring through practise with AI, I developed a number of
different speculative works to reflect through what was in reach of me
with AI. These ranged from situated community made datasets that
questioned the normalisation and isolation of indexes, as well as
interactive bits of code that made accessible how AI algorithms segment
space. The type of AI that I particularly oriented towards though and
got to know through practice was that of Reinforcement Learning (RL).
This was partially due to deepminds work with ``mastering'' Go and other
closed system games (Silver et al.~2016; 2017), and as this made room to
hype up to be the part of RL/AI that could encounter new things, the
unknown, or be ``creative''. It was also because it could be used to
train multi-agent environments and I wanted to explore how to form
environments of collaboration and organisation within these terms. I
went on to form works with RL, with the main one trying to configure
these ``agents'' so that the group of them could communicate to one
another through abstract patterns and emerge their own visual language
and representation of their environment together. These came out okay,
but I ended up desisting from working with them for a few reasons. The
main one was after a while I realised that the logics of the RL
algorithm that was supposed to enable AI to encounter the unknown, and
within AGI imaginaries cure every yet unknown problem, was in simple a
penal logic. This meant that the only feedback was to punish or reward
the agent to reinforce it to form a ``policy'' for that environments
task. It was also well noted to only work well in time sensitive
environments if you keep giving a constant punishment to the agents at
every step. In practice this system was very uncreative, the agent was
mostly not understanding what you wanted it to do and always trying to
trick the rules you laid out to just get the max points. When working in
this relationship especially it I tried to figure out a relation and
dynamic where I did not have to ``master'' or train them to get them to
learn anything. Another reason is that I felt that a lot of this
practice that I had done was very closed off and hard to collaborate
with others through. Here I realised that for me to use AI critically
and sustainably in any of my local community contexts was fairly far out
of my reach at the time. None of this made this technology either
accessible or desirable to me after a while. I slowly stopped to work
with these AI agents in the end, as there seemed no way to collaborate
with them without executing, mastering or punishing them, and the agents
always seemed to want to run away or escape when we came into contact
under these terms.

Another more direct explanation of this is again from D.A.I.R., but this
time from an episode of the \emph{Mystery AI Hype Theater 3000} podcast
by linguist Prof.~Emily M. Bender and sociologist Dr.~Alex Hanna. In
this podcast they often playfully take apart AI papers and their
critical underpinning. I bring this podcast up though as I want to focus
on Bender's critique of LLM's in the \emph{Can Machines Learn To Behave?
Part 3 (Hanna and Bender 2022)}. As she puts it\footnote{``Um but also
  then he goes on to say "Real insight began to emerge with word2vec"
  which is one of these first sort of neural approaches to language
  modeling. Um and it's like no all of the work in linguistics prior to
  2013 that's looking at at the relationship between form and meaning,
  all the work on distributional semantics before then, none of that is
  real insight? Real insight is when the engineers come in and throw
  their mathy math at it? . . . . I don't think so. . .'' (Hanna and
  Bender 2022)}, there are so many ways of modelling language that
exist, and yet they try to fit it all through this one single algorithm
that is based on eugenics, not overule any prior histories and practices
of language or linguistics. In a similar orientation I realised that yes
I might be able to organise with my community through AI and to fold up
our issues into abstract eugenics logics to automate our decisions, but
in doing this we lose the capacity and room to emerge the many other
logics and algorithms that we can interdependently and locally manifest
ourselves together from our divergent backgrounds. So as I disorient
away from this background of a cured future through automation I ask how
with the collaborations of this thesis I can make space for us to move
beyond these normalising practises of computing that prescribed
generalised logics of organisation onto our community? Through this
thesis' research and its inquiries I felt out these Configure-Able
methods as a way to form a response to this question and inquire into
how communities can come together to configure out their social and
technical infrastructures and politics locally.

Just to be clear this is not a total refusal of AI or to say that
algorithms are defined by their origins. It is instead to acknowledge
these histories and the limits of their logics when we do work with
them. With this crip relation to technology, it is to be okay with
technologies messiness, but to actively take care around/within their
limits. For instance in my inquiries, it is to be okay with the AI
speech to text algorithm that we have for now, but to both orient
towards community driven and joyous live captions Like I did in my
inquiries, but also to centres stenographer/transcription knowledges
within automated captioning technologies like Louise Hickman's cripping
of AI (\emph{Louise Hickman} 2021). For Laura Forlano as well, is to be
okay with the automated AI insulin pen (2023) and it's buggines, but in
this to again centre points of impact and the person's embodied
experience to these technologies in our critique of them.

\hypertarget{chapters}{%
\subsection{Chapters}\label{chapters}}

In this section I will give an overview of the chapters that make up
this thesis.

\hypertarget{disability-justice-and-life-affirmation-flipping-the-table}{%
\subsubsection{Disability justice and life affirmation flipping the
table}\label{disability-justice-and-life-affirmation-flipping-the-table}}

In
\href{../../01_Disability_justice_and_life_affirmation_flipping_the_table/01_Disability\%20justice\%20and\%20life\%20affirmation\%20flipping\%20the\%20table.md}{01\_Disability
justice and life affirmation flipping the table} I form the critical
framing of my research. The fields I work across here are Queer
Feminism, crip theory, and Feminist STS. I initially align with Sarah
Ahmed's \emph{Queer Phenomenology} (2006) to think through her concepts
of \emph{orientation}, \emph{sedimentation} and the \emph{bodily
horizon.} I do this to think around what is in reach, and how inherited
relations can orient me so that my contingent (other queer) desires are
out of reach. I also surface her \emph{soft body} and the ways it make
room for me to think about the impressions I make on others (and them on
me) and the technologies we stay in touch with. With Kafer's concepts I
aim to highlight the situated ways of approaching technologies through
social relations, as well as the capacity to mutually affect and
transform them and the ways I practice them through radical Queer and
crip politics.

I then orient closely to crip theory to feel out the radical politics
that I want to put in touch with these bodies of technoscience, network
infrastructuring and collective organising. To do this I take up a
reading of Alison Kafer's \emph{Feminist, Queer and crip}, (2013) among
other works, to emerge my own understanding of her Political/Relational
Model, crip time, and a crip intersectionality. With these I highlight
an understanding of crip theory as a place that has the capacity to
disorient our inherited and normalised organisation and networking
practices through the radical capacities of interdependence, care and
critical access. With Kafer I also take up a close reading of her figure
of the disabled body in technology, or the cyborg, stepping out of line
to be a situated body and horizon to map and configure technologies on
their own terms.

With Kafer's later \emph{Manifesting Manifestos} (2023) and her
\emph{Health Rebels} (2017) talk, I bring about the crip manifesto and
manifestings as a practice and framing for this research. As Kafer
describes it, how can we form a crazed practice of crip manifestos, each
imperfection, flaw and friction offering each iteration orientation
towards a future crips need and desire now. Here I also touch on Aimi
Hamraie and Kelly Fritsch's \emph{crip Technoscience Manifesto} (2019)
to orient this crazed crip manifesting towards STS and to inquire more
closely into the social and technical practices of infrastructures.
Their manifesto importantly for this work offering up how friction is
felt and made felt by others through the direct action of crip
technoscience. With them I take this up in my research to ask how we can
centre the knowledge of people at the sites of impact within the
infrastructures that not only support their communities but affirm their
lives.

With this chapter I aim to offer up this approach of a crazed crip
manifesting of manifestos as a way to orient my research away from the
inherited and sedimented norms of collective infrastructures. Through
these practices of manifesting disorientations I aim to bring in reach a
practice of community network infrastructuring that draws from crip
Theory and a version of STS that is sensitive to the inaccessibility of
roles, language and politics that science and technology normalise.

\hypertarget{crip-tic-of-vignettes}{%
\subsubsection{Crip-tic of Vignettes}\label{crip-tic-of-vignettes}}

\href{../../02_Crip-Tic_of_Vignettes/02_Crip-Tic_of_Vignettes.md}{02\_Crip-Tic\_of\_Vignettes}
accompanies the last one as a way to bring in the embodied feelings and
knowledges of the ableist relations that I theoretically explained in
the last chapter but have also experienced during my PhD and wider work.
I do this as it is important for my research around disability to be
situated at my table and contextualised from the experiences I have had.
This chapter aims to bring in much of the background and labour that is
often negated or hidden from view of the research table. It refuses to
hide these Queer and crip relations, but also to acknowledge my bodily
horizon as a gender queer disabled white middle class person, and make
that experience and context somewhat knowable in this research.

The crip-tic itself plays on the fractal scales of a tic or stim that is
discussed in \emph{Get The Frac In!} (Maier et al.~2020), unpredictable,
indefinite and recursive. This also plays with the scales of Tych
(diptych, triptych, etc.) in painting, where different paintings are
held together in indeterminate compositions. I work with this
composition of a crip-tic to offer up insights into some of the bodies I
have come into contact with during this research. The crip-tic also aims
to be left open to further Tics and indeterminate instances of crip
experience that go well out of my reach and make room for the readers to
add their own context. The very crip tics that I surface here are a
number of tables and normalised relations I have encountered within the
UK institutional context where I am based, and which I go through below.

The first table I come to is the \emph{Research Table(chap)}, where I
reflect on the frictions I felt when I was taken up as a crip research
object by an emerging X Digital Futures Institute. Here I highlight the
ways these researchers, and the norms of research they were operating
with, made no room for intimacy with the radial politics and practices
of crip studies, and in practice infantilised and undermined the
possibilities of these politics in action. In reflecting on these
frictions I aim to flip the table on how this research table, like many
before it, does not have the capacity for, and is not yet in reach of
the sorts of radical intimacy and access that crip practices demand.

I then go on to feel out the second table of the \emph{Computing Table
(chap)}, where I reflect on being in relation to the ex-cell table of
institutions. Here I orient it through both being an access and care
requester towards my own medical needs, as well as being a teacher and
care giver within institutions. Both of these orientations reflect on a
computing table that is configured to not make room for people's voices
or needs. These tables are instead oriented to extract specific figures
and data from people to hold them in place in their determined ex-cell.

The third crip-tic orients my most recent experiences of the
\emph{Operating Table (chap)} and institutional medical model of care.
This tic reflects on how I was treated when I had a flair up with my
health condition and tried to get care from these institutions. It
offers up an insight into the sick crip times that are currently
operated by these institutions in the UK and how they are nowhere near
in reach of offering me (or others) the care I knew I needed. This
inquiry and encounter was a reminder that we are often left off worse by
the operations and treatments normalised by these tables of
institutional ``care''. This reflection aims to highlight how the
medical model exists in my experience and contextualises it to the
current UK political climate of living in the aftermath of austerity
cuts to social care.

The final table I come to is the crip table, where I reflect on my
experiences of coming to tables and rooms of disability justice and life
affirmation. This was a wide range of things but I just touch upon some
workshops by M.E.L.T. called \emph{Rest Assured} that I attended, as
well as a day-ong symposium by Healing Justice London (HJL),
Transformational Governance Collective and Beyond the Rules' seminar
called \emph{Life Affirming Organisational Practices}. In both of these
experiences I reflect on how being in touch with both their practices of
access and capacities for people's needs induced a sense making of
\emph{guilt}. With this I realised that this feeling of guilt came from
the lack of these capacities that I had in myself but also in the
relations I had in reach at that time. This sense making also oriented
me to know, feel and figure out what sort of access, affirmation and
change I wanted to orient towards within my collective social and
technical practices.

I poetically sum up this chapter, aiming to bring these tables I am in
touch withwith those in theory and giving a material background of these
crip-tics that will be flipped. This is also left open to make room for
them and this research to be disoriented by both the reader's own
situated experiences, as well as those of the community and relational
tables I approach throughout the rest of this research. These two first
chapters make aim to offer up a bed for this research to then lay its
disobedient methodology comfortably atop of. With this making room for a
methodology that question the capacities, scales and actions I can make
room for with these crip politics and access knowledges in practice

\hypertarget{disobedient-action-research-cycles}{%
\subsubsection{Disobedient Action Research
Cycles}\label{disobedient-action-research-cycles}}

\href{../../03_Disobedient_Action_Research_Cycles/sections/03.00_Disobedient_Action_Research_Cycles.md}{03.00\_Disobedient\_Action\_Research\_Cycles}
covers my methodology for the research where I define my Disobedient
Action Research approach which is the methodology I enacted to feel out
the Configure-Able methods mentioned earlier. Disobedient Action
Research stems from Action Research (AR), which is a type of research
that centres practitioners as knowledge creators through doing, making
and being in relation to the material world. Disobedient AR specifically
emerges from a type of AR called First Person AR, which Judi Marsh
established with her notion of Living Life as Inquiry (LLI) (Gearty \&
Marshall, 2021; Marshall, 1999, 2004). Living Life as Inquiry and First
Person AR orient to situate themselves further within the role of the
practitioner as a knowledge maker. It does this by making room for the
practitioner to bring their background, context and embodied feelings
into their research practice (otherwise known as their first person
perspective). This offers a type of research where we have the capacity
to acknowledge where the research is coming from contextually and
relationally, as well as emotionally.

Disobedient Action Research, which TITiPI as well as Possible Bodies
(Pritchard, Rocha, and Snelting 2021; TITiPI, Ango, and Lepage 2022)
established, in some ways extends this orientation, but in other ways
troubles it. Disobedient Action Reseach extends it through making room
for us to do First Person Action Research interdependently. This moves
the knowledge making process from being within the self, to in between
the collective embodied selves. Disobedient Action Research also
troubles First Person Action Research by making demands of direct action
and change. Even though First Person AR calls for ``change agents'', it
is Disobedient AR that orients to question the limits change agents can
make. Specifically in the context of computer science and open source
software community Possible Bodies ask questions like what of the big
issues we feel around computing can actually be broken down and squeezed
like (code) bugs? And how can we give feedback and get feedback from
places where it is purposefully lacking?

I chose to use Disobedient Action Research as it aligns well with my
existing practice as a creative developer and community practitioner. It
also makes ample room for my context of being a crip researcher, one
where I question the scales and limits of politics and their relations.
In this it questions what participating within their defined limits
offers us, and for us to inquire into how to make change around, within
and to those limits. Disobedient AR also makes room for me to work with
the collectives and communities that I do, and to try to feel out our
collective, local and embodied refusal of Big Tech. Disobedient AR as a
methodology in practice also made room for me to develop my own
Configure-Able methods, which is what I go on to emerge throughout the
rest of this thesis.

\hypertarget{configure-able-methods}{%
\subsubsection{Configure-able methods}\label{configure-able-methods}}

\href{../../04_Configure-able_Methods/04_Configure-Able\%20Methods.md}{04\_Configure-Able
Methods} is one of the first actions of manifesting through inquiry
within this thesis. In it I start to lay down the foundational inquiries
into Configure-Able methods by cripping Science and Technology Studies
(STS) methods. I start by forming an accessible genealogy of
configuration working with Celia Lury et al.'s chapter on \emph{Figure,
Figuring and Configuration} (2022). This genealogy moves from figures as
abstract stories, roles and statistics, that we imagine and represent
society through, to configuring how these figured plans and stories are
brought into material action. I specifically highlight the work of Lucy
Suchman's \emph{Configuration} (2012) as it is both an accessible and
quality discussion of a feminist approach to configuration. I
particularly take her up as in her essay works with configuration to
critique the data management and organisational infrastructure projects
of medical care institutions, their supporting politics, and the
development of their future care infrastructures.

With this genealogy and with Suchman's (ibid) example I inquire into how
configuration can be disoriented when put in touch with crip theory?
What practices and capacities does a crip Configuration give room for?
To do this, I initially set up a framing with Aimi Hamraie's critique of
the figure of the user within disability design history and its roots
within eugenic industrial design logics that reinforce and shape soft
bodies to hard systems. With Hamraie's critique I orient to question how
I can challenge the role of the user by distributing the expertise
through collective interdependent inquiry. To do this moving to
practices of access knowledge formed through design frictions. To do
this I reflect on a workshop I ran with NEoN digital\footnote{https://neondigitalarts.com/}
in Dundee who are a community based organisation focusing on digital
access. These workshops in themselves oriented network accessibility,
collective action and Feminist server practices. My inquiry approaches
these workshops through a first person reflection and focus on my
experiences of configuring access for the first time for an event like
this, and how I tried to do this (where possible) through interdependent
joy, and impact centred care. In this inquiry reflecting on how
distributing the role of the expert within my collaborative group made
room for us to challenge the capacity of configuration as access making.
In this move disorienting the centralised and prescribed plans of
normative configuration through the points of friction and impact we
felt as a group together.

I then return to Suchman's \emph{Configuration} (2012) to bring it into
contact with this inquiry and these experiences. I do this by taking up
disability in her critique where it has been left abscent and invalid
from its own dialogues. By giving feedback as a disabled voice to her
analysis of the configuration of medical care systems and their politics
I imagine what capacities the people with disability at the site of
impact of these systems might want from configuration. Here I try to
make room to figure out how to uplift their capacity to communicate
their needs, feelings and agency through Configure-Able methods. This
again very disobediently poses Configure-Able as a question and a
prompt, so I can ask how Configure-Able is this relation, infrastructure
and politics? Am I supported by it? Or held in place? And how can my
collaborators and I orient around the frictions we feel and make room
for them to inform our collective configuration practices.

This chapter builds up these methods for the next two sections to work
through in practice and situate them to be made knowable within
different contexts, collaborations and practices.

\hypertarget{in-configure-ability}{%
\subsubsection{In-configure-ability}\label{in-configure-ability}}

\href{../../05_In-Configure-Ability/05_In-Configure-Ability.md}{05\_In-Configure-Ability}
covers one of the inquiries into situating Configure-Able methods within
embedded collective practice. To do this I collaborated with In-grid,
which I helped to set up in 2020. In this context, I worked with the
collective to situate Configure-Able methods within our practices to
make roo m to figure out both how we wanted to organise ourselves and
our digital network infrastructure together. In doing this, we also
queried how we were organised by the inherited relations of our
pre-existing social and technical infrastructures that we came together
through. To inquire into this with In-grid, I have been working through
Disobedient AR to practise how these Configure-Able methods can come
into being over a reasonable period of time (2+ years) and within an
embedded intersectional community. To reflect on this collaboration I
undertook research with a focus group, where In-grid members
collectively figured out our organisational practises together and
thinking through how we were oriented and flexed by a hard system, and
how we refused these normalised tables and relations, to feel out our
own . I follow this up with a first person AR reflections around
In-grid's technical knowledge sharing practices, informing the group
both with Kelsie Acton's notion of \emph{Plain Language} (2023) to make
room for us to reorient through access. I also work with In-grid to
develop the practising protocols workshops for 4S/EASST and which I went
on to run and reflect on through an internal workshop with members where
we configured out our network infrastructures together. This internal
workshop also lead to us writing and publishing In-grid's itteration of
the \emph{Feminist Server Manifesto} (Constant 2015), called the
Femfester which was published in the first edition of \emph{Artist
Running Datacentres} jounral (Simms et al.~2024).

This chapter aims to coalesce the many scales that I have been
manifesting Configure-Able methods within my collaborations with
In-grid. By situating these methods in an intersectional collective I
offer up how critical access can not only be an extra we add on to our
social and technical practices but a thing we can orient and action
directly from in generative ways. In this move it shares how in this
collaboration we made room to configure and plan together from the sites
of friction and impact that we felt together. This room for collective
access and the practices that emerged transformed the roles of users
through the validation of their expertise and making the skills and
capacities to enact these feelings within reach.

\hypertarget{a-cozier-configure-ability}{%
\subsubsection{A Cozier
Configure-Ability}\label{a-cozier-configure-ability}}

\href{../../06_A\%20Cozier\%20Configure-Ability/06_A\%20Cozier\%20Configure-Ability.md}{06\_A
Cozier Configure-Ability} reflects on a second inquiry into
Configure-Able methods. In this context I wanted to inquire into how I
could start to form a crip centred network practice. To do this I formed
the Cozy-Cloud as a server space that was made for and by disabled
people. This server acts as wiggle room for collaborators and me to try
to configure out what life affirming network infrastructures might feel
like together. This inquiry into a Configure-Able crip cloud is
developed through both manifesting cripped technical infrastructure, but
also the social relations of interdependent care and practices of design
friction that were needed to not only produce but to maintain this crip
network infrastructure. In this inquiry I surface two artistic works
that share how these localised crip centred manifestations of
Configure-Able methods have come into being.

The first work explores my manifesting of the intentions of the server
as I initially made it. These intentions were made through developing a
set of sticker and alt text pairs. This work by centring alt texts as
medium configures access to the front of their composition to highlight
the capacitythat access practices already have. These capacities of
access in the alt texts both lay in them being able to communicate the
images content to wider audiences with sensory differences, but also
makes the alt text accessible within reduced internet bandwidth that
cannot support images. Through these sticker and alt text pairs I
poetically manifest a set of cozy intentions for the Cozy-Cloud to
emerge from. Each of these forms their own multi-media orientation for
what this server is intending to wiggle crip politics within. These
intentions in message orient towards forming network infrastructures
that do not rely on imaginaries of infinite growth and exhaustive
relations. They intend to do this by making room for computational
practices and network infrastructures that consciously move away from
their militaristic origins and imaginaries. Alongside these intentions I
also manifested the Cozy-Cloud relaxation space for people to be in
reach of this crip server infrastructure and its cozy intentions in a
relaxing calm space. To emphasise this multi-sensory and media set of
intentions I also made cozy fluffy cushions, a printable easy read
version of the cozy intentions and physical versions of the stickers, so
that those visiting can interact and feel these intentions through in a
way that is comfy and accessible to them.

The second work I surface in this chapter is that of the En-crip-ing
Time, which was a collaboration with Marianna Marongoni\footnote{https://marianamarangoni.com/}.
This work was our inquiry into how to crip both the time and intimacies
of the Cozy-Cloud server and its relations to network practices. To do
this we configured out our own poetic performance of cron jobs
(deterministic and routine maintenance jobs on Unix operating systems)
as chronic jobs (indeterminate, indefinite care and maintenance through
intimacy). To make room for crip times here we hacked multiple bits of
code and syntax to make wiggle room within the existing infrastructure
and limits of network time for our poetic crip times. In this work we
also highlight the ways we made the infrastructure together, and how we
made room for each other and our needs so that this crip infrastructure
could come into being. I also touch upon the essay that we wrote for
Brand-New-Life about the work (Simms \& Marangoni, 2025), and where we
try to blur the boundaries, variability and formats of academic texts to
both hold technical dialogues, but also the very crip feeling Marianna
and I had around them.

I conclude the chapter by both reflecting back at these projects and how
they have transformed network practices by bring them in touch with crip
bodies, tables and politics, but also to orient forward to the futures
that the Cozy-Cloud is approaching. This future is one of slowly
building up more collaborations and infrastructures to slowly feel out
what accessible and configure-able crip network practices can be.
Whatever may lay in the indeterminate future of the Cozy-Cloud, I know
we will aim to carefully disorient the limits and futures of these
network infrastructures together right now!

\hypertarget{conclusion}{%
\subsection{Conclusion}\label{conclusion}}

This thesis and its inquiries go on to shape in action what
Configure-able methods can be in these localities. With these inquiries
into Configure-able methods in action I ask how can they make room for
us to orient from the site of impact within configuration? How can we
make room to disorient our plans from the frictions felt in them? And
how can we centre crip theory and critical access not only in the ways
we organise and care for our communities, but the ways this can shape
these hard systems we feel to orient toward the life affirming
infrastructures we know we need?
